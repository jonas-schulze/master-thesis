\usepackage{mathtools}
\usepackage{xparse,xspace}
\usepackage{booktabs}
\usepackage{csquotes}
\usepackage[shortcuts]{glossaries}
\usepackage[binary-units]{siunitx}[=v2]
\usepackage{listings}

% inspired by: https://tex.stackexchange.com/a/212794/95100
\lstdefinelanguage{julia}{%
  morekeywords={abstract,begin,break,case,catch,const,continue,do,else,elseif,%
      end,export,false,for,function,immutable,import,importall,if,in,%
      macro,module,otherwise,quote,return,struct,true,try,type,typealias,%
      using,where,while},%
  sensitive=true,%
  morecomment=[l]\#,%
}[keywords,comments]

\lstdefinestyle{julia}{%
  language=julia,
  basicstyle=\small\ttfamily,
  commentstyle=\color{black!15},
  keywordstyle=\fontseries{b}\selectfont,
  numberstyle=\scriptsize\sffamily\color{black!15},
  numbers=left,
  xleftmargin=20pt,
  % use special julia comments as range markers: #== text ==#
  rangeprefix=\#\=\=\ ,
  rangesuffix=\ \=\=\#,
  includerangemarker=false,
  % add latex labels using #* \label{line:...}
  escapeinside={\#*}{\^^M},
}

\lstdefinestyle{remark}{%
  language=,
  numbers=none,
  xleftmargin=0pt,
  basicstyle=\small\ttfamily\color{black!15}, % should match transparency of covered
}

\lstset{
  columns=flexible,
  style=remark,
}

\usepackage{tikz}
\usetikzlibrary{positioning}
\usetikzlibrary{graphs,arrows.meta}
\tikzset{>=Stealth}

\sisetup{
  round-mode = places,
  round-precision = 2,
}
