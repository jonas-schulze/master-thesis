\newcommand{\N}{\mathbb{N}} % Natural numbers
\newcommand{\Z}{\mathbb{Z}} % Whole numbers
\newcommand{\Q}{\mathbb{Q}} % Rational numbers
\newcommand{\R}{\mathbb{R}} % Real numbers
\newcommand{\C}{\mathbb{C}} % Complex numbers
\newcommand{\F}{\mathbb{F}} % Arbitrary Field
\newcommand{\K}{\mathbb{K}} % Arbitrary Field

\newcommand{\Cneg}{\C_-} % Negative Half Plane

\newcommand{\im}{i} % imaginary unit

\newcommand{\onehalf}{\tfrac{1}{2}}

% matrices
\newcommand\Cnn{\C^{n\times n}}
\newcommand\Rnn{\R^{n\times n}}
\newcommand\Rnr{\R^{n\times r}}
\newcommand\Rnk{\R^{n\times k}}
\newcommand\Rkk{\R^{k\times k}}
\newcommand\nnz{\operatorname{nnz}}
\renewcommand\vec{\operatorname{vec}}
\DeclareMathOperator{\colspan}{span}
\DeclareMathOperator{\orth}{orth}
\DeclareMathOperator{\rank}{rank}
\DeclareMathOperator{\diag}{diag}
\newcommand\MP{\dagger} % Moore Penrose pseudo-inverse

% transpose and conjugate/Hermitian transpose:
\newcommand\conj[1]{\overline{\optional{#1}}}
\newcommand\T{T}
\newcommand\HT{H}

% Rosenbrock
\newcommand\Ham{\ensuremath{H}}
\newcommand\Ricc{\operatorname{\mathcal R}}
\newcommand\Jac{\operatorname{\mathcal J}}

 % ADI
\newcommand\Aip{\mathop{H_k^+}}
\newcommand\Aim{\mathop{H_k^-}}
\newcommand\Aipm{\mathop{H_k^\pm}}
\newcommand\Aiip{\mathop{V_k^+}}
\newcommand\Aiim{\mathop{V_k^-}}
\newcommand\Aiipm{\mathop{V_k^\pm}}
\newcommand\Cayley{\mathop{\mathcal{C}}}
\newcommand\Aipinv{\mathop{(\Aip)^{-1}}}
\newcommand\Aiipinv{\mathop{(\Aiip)^{-1}}}
\newcommand\Lyap{\operatorname{\mathcal L}}

% usage: \{2x\given x\in\N}
\newcommand{\given}{\mid}

% usage: \Set[\big]{2x \given x\in\N}
\newcommand\SetSymbol[1][]{%
  \nonscript\:#1\vert
  \allowbreak
  \nonscript\:
  \mathopen{}}
\DeclarePairedDelimiterX{\Set}[1]{\lbrace}{\rbrace}{%
  \renewcommand\given{\SetSymbol[\delimsize]}% this effect is local only
  #1%
}

% personal taste:
\let\epsilon\varepsilon
%\renewcommand{\to}{\longrightarrow}
%\renewcommand{\mapsto}{\longmapsto}
%\renewcommand{\gets}{\longleftarrow}
\renewcommand{\Re}{\operatorname{Re}} % real part of a complex number
\renewcommand{\Im}{\operatorname{Im}} % imaginary part

% some more delimiters:
\NewDocumentCommand{\optional}{m}{\ifblank{#1}{\,\cdot\,}{#1}}
\DeclarePairedDelimiterX{\abs}[1]{\lvert}{\rvert}{\optional{#1}}
\DeclarePairedDelimiterX{\norm}[1]{\lVert}{\rVert}{\optional{#1}}
\DeclarePairedDelimiterX{\scalar}[2]{\langle}{\rangle}{\optional{#1},\optional{#2}}
\newcommand{\card}{\abs}

% integration:
\NewDocumentCommand{\intd}{m}{\,\textup{d}#1}
\newcommand\dt{\intd{t}}

% differentiation:
\NewDocumentCommand{\pdiff}{mm}{\frac{\partial #2}{\partial #1}}
\NewDocumentCommand{\diff}{mm}{\frac{\mathrm{d} #2}{\mathrm{d} #1}}
