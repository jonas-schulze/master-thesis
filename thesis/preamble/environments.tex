% Remember: use \label{thm:...} for theorems, propositions, lemmata, etc.
% This way the type of an environment can easily be changed.

\declaretheoremstyle[
  spaceabove = \topsep,
  spacebelow = \topsep,
  postheadspace = 0.5em, % matches proof
  headfont = \usekomafont{captionlabel},
  notefont = \normalfont\sffamily,
  bodyfont = \normalfont,
  qed = \ensuremath{\triangle\hspace*{-.1ex}}, % bring it a little bit closer to the edge
]{myplain}

\declaretheoremstyle[
  spaceabove = \topsep,
  spacebelow = \topsep,
  postheadspace = 0.5em, % matches proof
  headfont = \itshape,
  numbered = no,
  qed = \ensuremath{\triangle\hspace*{-.1ex}}, % bring it a little bit closer to the edge
]{myremark}

\renewcommand{\qedsymbol}{\ensuremath{\square}} %TODO: still not bold enough compared to \triangle

\declaretheorem[style=myplain, numberwithin=chapter]{lemma}
\declaretheorem[style=myplain, numberlike=lemma]{corollary}
\declaretheorem[style=myplain, numberlike=lemma]{proposition}
\declaretheorem[style=myplain, numberlike=lemma]{theorem}
\declaretheorem[style=myplain, numberlike=lemma]{hypothesis}

\declaretheorem[style=myplain, numberlike=lemma]{definition}
\declaretheorem[style=myplain, numberlike=lemma]{example}

\declaretheorem[style=myremark]{remark}


% ******************************************************************************
% Listings

\setlist[enumerate]{font=\normalfont}
\setlist[enumerate,1]{label=(\roman*)}
\setlist[enumerate,2]{label=(\alph*)}
\setlist[enumerate,3]{label=(\arabic*)}


% ******************************************************************************
% Source Code

\lstset{
  basicstyle=\small\ttfamily,
  numberstyle=\footnotesize\sffamily,
  numbers=left,
  % use special julia comments as range markers: #== text ==#
  rangeprefix=\#\=\=\ ,
  rangesuffix=\ \=\=\#,
  includerangemarker=false,
  % add latex labels using #* \label{line:...}
  escapeinside={\#*}{\^^M},
}

