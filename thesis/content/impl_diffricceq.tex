\section{DifferentialRiccatiEquations.jl}
\label{sec:impl:DRE}

This section describes the general properties of the \julia{DifferentialRiccatiEquations.jl} package.

\begin{itemize}
  \item
    Type of compression (none, $LDL^\T$) is selected by dispatching on the data type of the initial value $X_0$/\julia{X0},
    \cf~\autoref{lst:impl:GDREProblem}.
    For a dense matrix \julia{X0::Matrix}, a dense solver will be used.
    For a \ac{LRSIF} \julia{X0::\LDLt}, a low-rank solver will be used.
  \item
    \julia{X::\LDLt} of rank $r$ consisting of $L \in\R^{n\times r}$ and $D\in\R^{r\times r}$ will be compressed,
    \cf~\autoref{alg:lowrank:compression},
    if the inner dimension becomes too big, $r \geq 0.5 n$,
    and every 20 modifications without compression,
    \cf~\autoref{sec:lowrank}.
    For now, this is not configurable.
  \item
    Usage of Sherman-Morrison-Woodbury formula, \cf~\autoref{sec:basics:smw},
    is hidden from \ac{ADI} via \julia{LowRankUpdate}.
\end{itemize}

\begin{lstlisting}[%
  float=t,
  caption={Definition and docstring of \julia{GDREProblem}},
  label={lst:impl:GDREProblem},
  escapechar=\%,
]
"""
Generalized differential Riccati equation

    E'%\.X%E = C'C + A'XE + E'XA - E'XBB'XE
    X(t0) = X0

having the fields `E`, `A`, `C`, `X0`, and `tspan`=`(t0, tf)`.
"""
struct GDREProblem{XT}
    E
    A
    B
    C
    X0::XT
    tspan

    function GDREProblem(E, A, B, C, X0::XT, tspan) where {XT}
        new{XT}(E, A, B, C, X0, tspan)
    end
end
\end{lstlisting}

\begin{figure}[t]
  \includegraphics[width=\textwidth]{figures/fig_ros2_K171.pdf}
  \caption{Trajectory of $K_{1,71}$ computed by \Ros{2}}
\end{figure}

\begin{figure}[t]
  \includegraphics[width=\textwidth]{figures/fig_ros2_error_lorwank_v_dense.pdf}
  \caption[Relative error of $K$ for LRSIF \Ros{2} vs reference solution]{%
    Relative error of $K$ between low-rank and dense \Ros{2} of the same step size $\tau$.
    This looks like a pole at $t=\SI{45}{\second}$.
    The actual error at $t=\SI{45}{\second}$ is not shown, as it is determined by the initial value, \ie it is zero.
    This more so hints at an inconsistent initial value of the \ac{DRE}.
    The general downward trend of the error for decreasing $t$ could be explained by the limit being an \ac{ARE},
    which doesn't depend on the initial value at all.
  }
\end{figure}

\begin{figure}[t]
  \includegraphics[width=\textwidth]{figures/fig_ros2_error_lorwank_v_ref.pdf}
  \caption[Relative error of $K$ for LRSIF \Ros{2} vs dense counterpart]{%
    Relative error of $K$ between low-rank \Ros{2} and dense reference solution computed with step size $\tau=\SI{10}{\milli\second}$
  }
\end{figure}

\begin{figure}[t]
  \includegraphics[width=\textwidth]{figures/fig_ros2_mean_error.pdf}
  \caption[Mean relative error of $K$]{%
    Mean relative error of $K$ on time interval $[\SI{43.2}{\second}, \SI{45}{\second}]$.
    The reference solution has been computed with dense \Ros{2} and step size $\tau=\SI{10}{\milli\second}$.
  }
\end{figure}

\begin{figure}[t]
  \includegraphics[width=\textwidth]{figures/fig_ros2_single_error.pdf}
  \caption[Relative error of $K(\SI{43.2}{\second})$ for LRSIF and dense counterpart]{%
    Relative error of $K(\SI{43.2}{\second})$ between low-rank and dense \Ros{2} of the same step size $\tau$
  }
\end{figure}
