\section{DifferentialRiccatiEquations.jl}
\label{sec:impl:DRE}

% Don't mention the inconsistency in tspan!  I was inspired by Norman's code,
% which received tspan sorted, i.e. in forwards time, and internally reversed
% it before and after solving.  Not knowing what I was doing, I thought
% accepting tspan in backwards time was a clever idea.  Now this package has to
% be used with backwards tspan, while having the equations stated in forwards
% time.
%
% https://gitlab.mpi-magdeburg.mpg.de/jschulze/DifferentialRiccatiEquations.jl/-/issues/6#note_13652

The general user interface looks like this:
\begin{lstlisting}[
  caption={[User Interface of \code{DifferentialRiccatiEquations.jl}]},
]
prob = GDREProblem(E, A, B, C, X0, tspan)
alg = Ros1()
sol = solve(prob, alg; dt=dt)
\end{lstlisting}
Besides \julia{Ros1()} and \julia{Ros2()},
which correspond to the algorithms described in \autoref{sec:ros},
there also is a \julia{Ros4()} solver.
It is based on \cite[Appendix~A]{Lang2017} and only supports the dense case.
The \julia{GDREProblem} type is parametrized on the type of the initial value \julia{X0},
by which the proper \ac{ALE} solver is selected via runtime dispatch on \julia{solve()}.
Dense initial values are represented by the built-in \julia{Matrix} type,\footnote{%
  \url{https://docs.julialang.org/en/v1.6/stdlib/LinearAlgebra/}}
and lead to a direct \ac{ALE} solver from \code{MatrixEquations.jl}~\cite{MatrixEquations}.
Initial values in \ac{LRSIF} are represented by the \julia{\LDLt} type,\footnote{%
  There are extensions for many editors to obtain \julia{\LDLt} via
  \julia{LDL\textbackslash\textasciicircum T} and then hitting the Tab key,
  \eg Vim, Emacs and VS Code.
  \cf \url{https://julialang.org/}
}
and lead to the iterative \ac{ALE} solver described in \autoref{sec:ADI}.

\paragraph{Representation of Low-Rank Matrices}

Suppose $X = LDL^\T$ and \julia{X = \LDLt(L, D)} are mathematical and Julia representations of the same objects,
$L\in\R^{n\times r}$, $D\in\R^{r\times r}$.
Then, the internal representation of~\julia{X} consists of lists~\julia{[L]} and~\julia{[D]},
such that updates may be performed lazily by merely collecting all the summands,
\cf \autoref{sec:lowrank}.
Compression of~\julia{X} is performed once these lists contain 10 elements or more,
or if the equivalent inner dimension is too large, $r \geq \onehalf n$.
This causes a compression every 10 iterations of the \ac{ADI} or earlier.
Currently, this is not configurable,
\cf \autoref{lst:impl:LDLt}.

\begin{lstlisting}[
  float=tp,
  caption={%
    [{Definition of \julia{\LDLt} and \julia{(+)}}]
    Definition of \julia{\LDLt} and addition \julia{(+)} of other \ac{LRSIF} objects.
    The compression, \julia{compress!()}, is described in \autoref{alg:lowrank:compression}.
  },
  label={lst:impl:LDLt},
  escapechar=\%,
]
struct %\LDLt%{TL,TD}
    Ls::Vector{TL}
    Ds::Vector{TD}
end

function Base.:(+)(Xs::%\LDLt%{TL,TD}...) where {TL,TD}
    Ls = TL[]
    Ds = TD[]
    for X in Xs
        append!(Ls, X.Ls)
        append!(Ds, X.Ds)
    end
    X = %\LDLt%{TL,TD}(Ls, Ds)
    maybe_compress!(X)
end

function compression_due(X::%\LDLt%)
    # If there is only one component, it has likely already been compressed:
    length(X.Ls) == 1 && return false
    # Compression is due every couple of modifications:
    length(X.Ls) >= 10 && return true
    # Compression is due if rank is too large:
    n = size(X, 1)
    r = rank(X)
    return r >= 0.5n
end

function maybe_compress!(X::%\LDLt%)
    compression_due(X) || return X
    compress!(X)
end
\end{lstlisting}

\paragraph{Representation of Shifted Matrices}

Recall that the Rosenbrock methods yield \ac{ALE} whose system matrices are sparse matrices shifted by a low-rank matrix,
\cf $\tilde A_n$ on \autoref{alg:ros1:smw} of \autoref{alg:ros1} and $\hat A_n$ on \autoref{alg:ros2:smw} of \autoref{alg:ros2}.
Within the \ac{ADI}, these matrices are updated by another sparse matrix,
\cf $A^+_k$ on \autoref{alg:ADI:smw} of \autoref{alg:ADI}.
Overall, taking \Ros{1} as an example,
the system to be solved with is
\begin{equation}
  \big(\big( A_n - \tfrac{1}{2\tau} E \big) + \alpha_k E^\T \big) - B K_n
  .
\end{equation}
This structure, however, is completely hidden from the \ac{ADI} code via
\begin{lstlisting}[
  caption={[Abstraction of shifted linear system between Rosenbrock method and \acs*{ADI}]},
  %caption={Representation of shifted linear systems},
  escapechar=\%,
]
%\~A% = LowRankUpdate(A - E/(2%$\tau$%), -1, B, K)
R = %\LDLt%(G, S)
lyap = GALEProblem(E, %\~A%, R)
solve(lyap, ADI())
\end{lstlisting}
where the \ac{ADI} adds the sparse $\alpha_k E^\T$ via \julia{(+)} without losing the \julia{LowRankUpdate} structure.
Its main feature, however, is solving the corresponding linear system via \julia{(\textbackslash)}
using the Sherman-Morrison-Woodbury formula,
\cf \autoref{sec:basics:smw}.
See \autoref{lst:impl:LowRankUpdate} for the basic definition of \julia{LowRankUpdate} and the aforementioned functions.

\def\mi{\textsuperscript{-1}}
\begin{lstlisting}[
  float=tp,
  caption={%
    [{Definition of \julia{LowRankUpdate} as well as \julia{(+)} and \julia{(\textbackslash)}}]
    Definition of \julia{LowRankUpdate},
    a lazy representation of the shifted linear system $A + \alpha^{-1} UV$,
    as well as addition \julia{(+)} of a sparse matrix,
    and left-division \julia{(\textbackslash)}
    using the Sherman-Morrison-Woodbury formula,
    \cf \autoref{sec:basics:smw}.
  },
  label={lst:impl:LowRankUpdate},
  escapechar=\%,
]
struct LowRankUpdate{TA,T,TU,TV}
    A::TA
    %$\alpha$%::T
    U::TU
    V::TV
end

function Base.:(+)(AUV::LowRankUpdate, E::AbstractMatrix)
    @assert issparse(E)
    A, %$\alpha$%, U, V = AUV
    LowRankUpdate(A+E, %$\alpha$%, U, V)
end

function Base.:(\)(AUV::LowRankUpdate, B::AbstractVecOrMat)
    A, %$\alpha$%, U, V = AUV

    FA = _factorize(A)
    A%\mi%B = FA \ B
    A%\mi%U = FA \ U

    S = %$\alpha$%*I + V * A%\mi%U
    S%\mi%VA%\mi%B = S \ (V * A%\mi%B)

    X = A%\mi%B - A%\mi%U*S%\mi%VA%\mi%B
    return X
end
\end{lstlisting}

