\subsection{Parareal Method}

\begin{frame}<-2>{Parareal Method}
\begin{bigpicturecols}
  \begin{block}{General Formulation \parencite{Lions2001}}
    For the IVP $\dot u = f(u)$ the method reads
    \begin{equation*}
      \left\{
      \begin{aligned}
        U^0_{n+1} &:= G(U^0_n) \\
        U^{k+1}_{n+1} &:= G(U^{k+1}_n) + F(U^k_n) - G(U^k_n)
      \end{aligned}
      \right.
    \end{equation*}
    where $U_0^0 := u(t_0)$. $U_n^k$ converges to $u(t_n)$ as $k\to\infty$.
  \end{block}
  \begin{itemize}
    \item
      Köhler, Saak, and Lang 2016:
      LRSIF formulation
  % GAMM Annual Meeting
  \begin{equation*}
    U^{k+1}_{n+1}
    =\vphantom{\Bigg[}\parbox{6cm}{$%
    \alt<1>{
      \colorldlt{\cola}
    + \colorldlt{\colb}
    - \colorldlt{\colc}
    }{
    \colorspacing % must be located in respective cell!
    \Bigg[
    \begin{matrix}
      \tallcmat{\cola} &
      \tallcmat{\colb} &
      \tallcmat{\colc}
    \end{matrix}
    \Bigg]
    \begin{bmatrix}
      \smallcmat{\cola} \\
      & \smallcmat{\colb} \\
      && -\smallcmat{\colc}
    \end{bmatrix}
    \begin{bmatrix}
      \widecmat{\cola} \\
      \widecmat{\colb} \\
      \widecmat{\colc}
    \end{bmatrix}
    }$}
  \end{equation*}
    \item
      Speed-up $\approx t_F/t_G$ for large $N$,
      where $0 \leq n \leq N$
  \end{itemize}
\column{\bigpicturewidth}
\bigpicture{6}
\end{bigpicturecols}
\end{frame}

%TODO: replace caption by parareal update formula?
\begin{frame}[plain]
  \setbeamertemplate{caption}[numbered]
  \frametitle{Parareal Example}
\begin{columns}
\column{0.65\textwidth}
  \begin{figure}
  \foreach \k in {0,1} {%
  \foreach \n [evaluate={\i=int(1+\n+\k*7)}] in {0,...,6} {%
    \includegraphics<\i>[width=\textwidth]{figures/parareal-anim/step-\k-\n.pdf}%
  }}%
  \includegraphics<15>[width=\textwidth]{figures/parareal-anim/step-2-5.pdf}%
  \includegraphics<16>[width=\textwidth]{figures/parareal-anim/step-2-6.pdf}%
  \includegraphics<17>[width=\textwidth]{figures/parareal-anim/step-3-5.pdf}%
  \includegraphics<18>[width=\textwidth]{figures/parareal-anim/step-3-6.pdf}%
  \renewcommand\thefigure{6.4} % number in thesis
  \caption{Parareal method applied to a linear ODE}
  \end{figure}
\column{0.35\textwidth}
  %\definecolor{rainbow1}{rgb}{0.5019608f0, 0.0f0, 0.5019608f0}
%\definecolor{rainbow2}{rgb}{0.0f0, 0.0f0, 1.0f0}
%\definecolor{rainbow3}{rgb}{0.0f0, 0.5019608f0, 0.0f0}
%\definecolor{rainbow4}{rgb}{1.0f0, 0.64705884f0, 0.0f0}
%\definecolor{rainbow5}{rgb}{1.0f0, 0.0f0, 0.0f0}

\newcommand\pararealU[2]{\draw (2*#2*\xshift,0) +(\yangle:2*#1*\yshift) node (U#1#2) {$U_{#1}^{#2}$};}
\newcommand\pararealG[2]{\draw (2*#2*\xshift,0) +(\yangle:2*#1*\yshift+\yshift) node (G#1#2) {$G(U_{#1}^{#2})$};}
\newcommand\pararealF[2]{\draw (2*#2*\xshift+\xshift,0) +(\yangle:2*#1*\yshift+\yshift) node (F#1#2) {$F(U_{#1}^{#2})$};}

\begin{tikzpicture}[diag/.style={out=45,in=180}]
  \footnotesize
  \def\xshift{11mm}
  \def\yshift{8mm}
  \def\yangle{90}
  % initial value
  \action<+->{\pararealU{0}{0}}
  % k = 0, coarse solutions
  \foreach \n [evaluate={\nprev=int(\n-1)}] in {1,...,5} {%
  \action<+->{%
    \pararealG{\nprev}{0}
    \pararealU{\n}{0}
    \draw [->] (G\nprev0) -- (U\n0);
    \draw [->] (U\nprev0) -- (G\nprev0);
  }}
  % k = 0, fine solutions
  \action<+->{%
  \foreach \n in {0,...,4} {%
    \pararealF{\n}{0}
    \draw [->] (U\n0) -- (F\n0);
  }}
  % k = 0, transform to ghost
  \action<+->{}
  % k = 1, coarse solutions
  \action<+->{%
  \pararealU{1}{1}
  \draw [->] (F00) -- (U11);
  }
  \foreach \n [evaluate={\nprev=int(\n-1)}] in {2,...,5} {%
  \action<+->{%
    \pararealG{\nprev}{1}
    \pararealU{\n}{1}
    \draw [->] (G\nprev1) -- (U\n1);
    \draw [->] (F\nprev0) -- (U\n1);
    \draw [->] (G\nprev0) to [diag] (U\n1);
    \draw [->] (U\nprev1) -- (G\nprev1);
  }}
  % k = 1, fine solutions
  \action<+->{%
  \foreach \n in {1,...,4} {%
    \pararealF{\n}{1}
    \draw [->] (U\n1) -- (F\n1);
  }}
  % k = 2, coarse solutions
  %\action<+->{%
  %\pararealU{2}{2}
  %\draw [->] (F11) -- (U22);
  %\foreach \n [evaluate={\nprev=int(\n-1)}] in {3,...,5} {%
  %  \pararealG{\nprev}{2}
  %  \pararealU{\n}{2}
  %  \draw [->] (G\nprev2) -- (U\n2);
  %  \draw [->] (F\nprev1) -- (U\n2);
  %  \draw [->] (G\nprev1) to [diag] (U\n2);
  %  \draw [->] (U\nprev2) -- (G\nprev2);
  %}}
  % k = 2, fine solutions
  %\action<+->{%
  %\foreach \n in {2,...,4} {%
  %  \pararealF{\n}{2}
  %  \draw [->] (U\n2) -- (F\n2);
  %}}
  % rest
  \action<+->{%
  \foreach \n in {1,...,4} {%
    \draw (F\n1.north east)+(40:1ex) node [rotate=35] {$\cdots$};
  }}
\end{tikzpicture}

\end{columns}
\end{frame}

