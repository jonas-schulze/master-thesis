\chapter{Alternating-Directions Implicit Method}

The \ac{ADI} method goes back to \cite{Peaceman1955}.
It has originally been designed to solve linear systems
$Ax=b$
for symmetric positive-definite $A\in\Rnn$
and applied to elliptic and parabolic \acp{PDE} using a finite-difference scheme.
In this chapter, after stating the classical parametrized intermediate-/two-step formulation,
we describe its properties as a one-step method.
Next, we will apply the method to \ac{ALE} in \autoref{sec:adi:ale},
and determine suitable parameters in \autoref{sec:adi:parameters}.
Finally, we will mention some alternative solvers for \ac{ALE}.

Suppose $A=A_1+A_2$ and $x\in\R^n$ is a solution to $Ax=b$.
Then it is easy to see that
\begin{align*}
  (A_1 + \alpha I_n)x &= b - (A_2 - \alpha I_n) x \\
  (A_2 + \beta I_n)x &= b - (A_1 - \beta I_n) x
\end{align*}
for any $\alpha, \beta \in\C$.
Define the short-hand notation
\begin{equation}
\label{eq:adi:shorthand}
\begin{aligned}
  \Aip  &:= A_1 + \alpha_k I_n &
  \qquad\qquad\qquad %FIXME
  \Aiip &:= A_2 + \beta_k  I_n \\
  \Aim  &:= A_1 - \beta_k  I_n &
  \Aiim &:= A_2 - \alpha_k I_n
\end{aligned}
\end{equation}
and convert the above into an iteration scheme
\begin{equation}
  \label{eq:adi:general2step}
  \left\{
  \begin{aligned}
    \Aip  x^{k+\frac{1}{2}} &= b - \Aiim x^k \\
    \Aiip x^{k+1}           &= b - \Aim x^{k+\frac{1}{2}}
  \end{aligned}
  \right.
\end{equation}
where the parameters $\alpha_k, \beta_k \in\C$ are free to choose for every iteration $k\in\N$.

\section{ADI as a One-Step Method}
\label{sec:adi:1step}

A solution to $Ax=b$ is, by construction, a fixpoint of the \ac{ADI} iteration~\eqref{eq:adi:general2step},
$x = x^{k+1} = x^{k+\frac{1}{2}} = x^k$.
Conversely, a fixpoint $\hat x := x^{k+1} = x^k$ is a solution to $Ax=b$,
provided that $\alpha_k + \beta_k$ is invertible.
Therefore, formulate the \ac{ADI}~\eqref{eq:adi:general2step} as a one-step scheme.

Obviously, $\Aip\Aim = \Aim\Aip$.
Thus, by \autoref{thm:adi:commuting-matrices} below, $\Aim\Aipinv = \Aipinv\Aim$.
Substituting the intermediate step $x^{k+\frac{1}{2}}$ into the final part of \eqref{eq:adi:general2step} leads to
\begin{align*}
  \Aiip \hat x
  &= b - \Aim \mathop{(\Aip)^{-1}} (b - \Aiim \hat x) \\
  &= b - \mathop{(\Aip)^{-1}} \Aim (b - \Aiim \hat x)
  .
\end{align*}
Multiplying by $\Aip$ and rearranging the terms then gives
\begin{equation*}
  \underbrace{%
  (\Aip \Aiip - \Aim \Aiim)
  }_{(\alpha_k + \beta_k) (A_1+A_2)}
  \hat x =
  \underbrace{%
  (\Aip - \Aim)
  }_{(\alpha_k + \beta_k) I_n}
  b
  .
\end{equation*}
Finally, multiplying by $(\alpha_k + \beta_k)^{-1}$ shows $A \hat x = b$.

\begin{lemma}[Commuting Matrices]
\label{thm:adi:commuting-matrices}
  Let $\alpha, \beta\in\C$ and let $A, B \in\Cnn$ commute.
  If $A$ is invertible, then $A^{-1}B = BA^{-1}$.
  If $(A+\alpha I_n)$ is invertible, then
  \begin{equation*}
    (A+\alpha I_n)^{-1} (B-\beta I_n)
    = (B-\beta I_n) (A+\alpha I_n)^{-1}
    .
  \end{equation*}
  Furthermore, for $M\in\Cnn$ it holds
  \begin{equation*}
    (A+\alpha M)^{-1} (A-\beta M)
    = I_n - (\alpha+\beta) (A+\alpha M)^{-1} M
    .
  \end{equation*}
\end{lemma}
\begin{proof}
  Multiplying $AB=BA$ with $A^{-1}$ from either side yields $A^{-1}B=BA^{-1}$.
  Furthermore, $AB=BA$ implies
  $
    (A+\alpha I_n) (B-\beta I_n)
    =
    (B-\beta I_n) (A+\alpha I_n)
  $.
  Applying the first claim yields the desired equality.
  The last claim follows from noting
  $A-\beta M = (A+\alpha M) - (\alpha+\beta)M$
  and multiplying by $(A+\alpha M)^{-1}$.
\end{proof}

More rigorously,
using the notation defined at the beginning of the chapter,
as a general one-step iteration the \ac{ADI} reads
\begin{equation*}
\begin{aligned}
  x^{k+1}
  &= \mathop{(\Aiip)^{-1}} \big( b - \Aim \mathop{(\Aip)^{-1}} (b - \Aiim x^k) \big) \\
  &= \mathop{(\Aiip)^{-1}} \underbrace{
    \Aim \mathop{(\Aip)^{-1}}
  }_{
    \mathop{(\Aip)^{-1}} \Aim
  }
  \Aiim x^k + \mathop{(\Aiip)^{-1}} \big( I_n - \underbrace{
    \Aim \mathop{(\Aip)^{-1}}
  }_{
    \mathop{(\Aip)^{-1}} \Aim
  }
  \big) b
  .
\end{aligned}
\end{equation*}
By \autoref{thm:adi:commuting-matrices},
$\Aipinv\Aim = I_n - (\alpha_k + \beta_k)\Aipinv$,
such that the iteration matrices for $x^{k+1} = M x^k + N b$ are given by:
\begin{equation}
\label{eq:adi:general1step}
  \begin{aligned}
    M &= (\Aip\Aiip)^{-1} (\Aim\Aiim) \\
    N &= (\alpha_k + \beta_k) (\Aip\Aiip)^{-1}
  \end{aligned}
\end{equation}

\todo{Find reference.}
Such an iteration converges iff $\rho(M) < 1$.
We restrict the analysis to commuting $A_1, A_2$.
In this case,
there exist parameters $\Set{(\alpha_k, \beta_k) \given k\in\N}$
such that the \ac{ADI} shows
\todo{Can I show this?}
superlinear convergence \cite{Beckermann2010}.

\begin{lemma}[Cayley Transformation]
\label{thm:adi:cayley}
  Let $A, M\in\Cnn$ and $\alpha, \beta \in\C$ with $-\alpha\notin\Lambda(A, M)$.
  The \emph{generalized Cayley transformation} of a matrix pair $(A,M)$ is defined by
  the rational matrix function
  \begin{equation*}
    \Cayley(A, M, \alpha, \beta) := (A+\alpha M)^{-1} (A-\beta M).
  \end{equation*}
  \todo[inline]{
    Furthermore, $\Lambda(A,M) \subset\C_- \implies \rho(\Cayley(A,M,\alpha,\conj\alpha)) < 1$,
    \cf \cite[Proposition~2.16]{Kuerschner2016}.
  }
\end{lemma}
\begin{remark}
  In the context of control theory,
  the last claim states that
  $\Cayley(A, M, \alpha, \conj\alpha)$ is d-stable if
  $(A, M)$ is c-stable.
\end{remark}
\begin{proof}
\end{proof}

\begin{lemma}
  Let $\alpha_k,\beta_k\in\C$ and $A=A_1+A_2$ be given.
  Define $\Aipm, \Aiipm$ according to \eqref{eq:adi:shorthand}.
  The \ac{ADI} iteration matrix $M$ as in \eqref{eq:adi:general1step} is a Cayley transformation.
  Furthermore,
  \begin{equation*}
    M = I_n - (\alpha_k + \beta_k) (\Aip\Aiip)^{-1} A
    .
  \end{equation*}
\end{lemma}
\begin{proof}
  For brevity, we omit the subscript $k$.
  Observe
  \begin{align*}
    \Aip\Aiip
    &= A_1A_2 + \alpha\beta I_n + \beta A_1 + \alpha A_2 \\
    &= A_1A_2 + \alpha\beta I_n - (\alpha-\beta)A_1 + \alpha(A_1 + A_2) \\
    \Aim\Aiim
    &= A_1A_2 + \alpha\beta I_n - \alpha A_1 - \beta A_2 \\
    &= A_1A_2 + \alpha\beta I_n - (\alpha-\beta)A_1 - \beta(A_1 + A_2)
  \end{align*}
  such that
  \todo{Check whether $-\alpha\in\Lambda(B, A)$.}
  $M = \Cayley(B, A_1+A_2, \alpha, \beta)$ for
  $
    B := A_1A_2 + \alpha\beta I_n - (\alpha-\beta)A_1
  $.
  The final equality follows directly from \autoref{thm:adi:commuting-matrices}.
\end{proof}

\todo[inline]{%
  Applying the above
  to an \ac{ALE}, $L = L_L + L_R$, leads to
  $
    X^{k+1} = X^k - 2\Re(\alpha_k) A_+^{-1} (A X^k + X^k A^\HT + W) A_+^{-\HT}
  $
  which looks pretty but doesn't seem to be beneficial.
  Is this worth mentioning?
}

If $A_1, A_2$ commute,
all the matrices $\Aip,\Aim,\Aiip,\Aiim$ commute.
By \autoref{thm:adi:commuting-matrices} this extends to their inverses.
Thus,
\begin{equation*}
  M =
  \Aipinv\Aim
  \Aiipinv\Aiim
  =:
  \Cayley(A_1, I_n, \alpha, \beta)
  \cdot
  \Cayley(A_2, I_n, \beta, \alpha)
  ,
\end{equation*}
\todo{Does this help with $\rho(M)<1$?}
\ie $M$ is even a sequence of Cayley transformations.

\begin{hypothesis}[Convergence of ADI]
\label{thm:adi:convergence}
\todo{Rephrase in terms of eigenvalues}
  If $A_1$ and $A_2$ are stable, then $\rho(M) < 1$.
\end{hypothesis}

\section{Application to \act{ALE}s}
\label{sec:adi:ale}

Consider the \ac{ALE}
\begin{equation*}
\label{eq:adi:ale}
  L(X) := AX + XA^\HT = -W
\end{equation*}
where $W = W^\T$.
The \Lyapunov operator $L = L_L + L_R$ naturally decomposes into
left-multiplication $L_L : X \mapsto AX$ and
right-multiplication $L_R : X \mapsto XA^\HT$.
Furthermore, $L_L$ and $L_R$ commute:
\begin{equation*}
  L_L \circ L_R = L_R \circ L_L = X \mapsto AXA^\HT
\end{equation*}
All the operators above are linear and act on $\Rnn$.
Apply the \ac{ADI}~\eqref{eq:adi:general2step} as a two-step method:
\begin{equation}
  \left\{
  \begin{aligned}
    (A + \alpha_k I_n) X_{k+\frac{1}{2}} &= -W - X_k (A^\HT - \alpha_k I_n) \\
    X_{k+\frac{1}{2}} (A^\HT + \beta I_n) &= -W - (A - \beta I_n) X_k
  \end{aligned}
  \right.
\end{equation}
The one-step formulation following \eqref{eq:adi:general1step} reads:
\begin{multline}
  X_{k+1} =
  (A + \alpha_k I_n)^{-1}
  (A - \beta_k I_n)
  X_k
  (A - \conj{\alpha_k} I_n)^\HT
  (A + \conj{\beta_k} I_n)^{-H}
  \\
  - (\alpha_k + \beta_k)
  (A + \alpha_k I_n)^{-1}
  W
  (A + \conj{\beta_k} I_n)^{-\HT}
\end{multline}
Choosing $\beta_k := \conj{\alpha_k}$
the above becomes Hermitean:
\begin{multline}
\label{eq:adi:ale1step}
  X_{k+1} =
  (A + \alpha_k I_n)^{-1}
  (A - \conj{\alpha_k} I_n)
  X_k
  (A - \conj{\alpha_k} I_n)^\HT
  (A + \alpha_k I_n)^{-H}
  \\
  - 2\Re(\alpha_k)
  (A + \alpha_k I_n)^{-1}
  W
  (A + \alpha_k I_n)^{-\HT}
\end{multline}
Define
$A^+_k := A + \alpha_k I_n$ and
$A^-_k := A - \conj\alpha_k I_n$.
For $X_k = L_k D_k L_k^\T$ and $W = GSG^\T$,
the \ac{ADI} can be stated in \ac{LRSIF} directly:
\begin{align*}
  \hat L_{k+1} &= \begin{bmatrix}
    (A^+_k)^{-1} A^-_k L_k &
    (A^+_k)^{-1} G
  \end{bmatrix} \\
  \hat D_{k+1} &= \begin{bmatrix}
    D_k \\
    & -2\Re(\alpha_k) S
  \end{bmatrix}
\end{align*}

\todo[inline]{%
  Exploit commutativity of $L_L, L_R$ to reorder ADI parameters.
  This way, $L_{k+1}$ is created from $L_k$ by augmenting a single block column
  instead of needing to multiplying $L_k$.
  Also, find reference to where does this idea came from?
}
...
and a compression $L_{k+1} D_{k+1} L_{k+1}^\T :\approx \hat L_{k+1} \hat D_{k+1} \hat L_{k+1}^\T$
following \autoref{sec:lr:compression}.
Taking the initial value $X_0 = 0$, the first iterate reads
\begin{align*}
  L_1 &= (A+\alpha_0 I_n)^{-1} G \\
  D_1 &= -2\Re(\alpha_0) S
  .
\end{align*}

\begin{remark}
  Let $\alpha\in\C\setminus\R$ and $A_+ := A+\alpha I_n \in\Cnn$.
  As a mapping of matrices,
  $L_R + \alpha I_{n^2} : \Cnn \to \Cnn$ maps $U$ onto
  $
    U A^\HT + \alpha U =
    U(A+\conj{\alpha} I_n)^\HT \neq
    U A_+^\HT
  $.
  Therefore, the notation is slightly more involved than in the previous section.
  For more details, refer to \autoref{sec:vectorization}.
\end{remark}

\begin{remark}
  \citeauthor{Lang2017}~\cite{Lang2017} formulates the \ac{ALE}~\eqref{eq:adi:ale} in terms of the
  right-multiplication $\tilde L_R :X \mapsto XA^\T$, $\tilde L_R \neq L_R$.
  The ADI~\cite[Equation~(2.23)]{Lang2017} uses a somewhat inconsistent second step.
  This leads to the impression that the ADI is a one-parameter ADI having $\tilde\beta_k = \tilde\alpha_k$
  instead of being a true two-parameter ADI using $\beta_k = \conj{\alpha_k}$.
  The one-step formulation \cite[Equation~(2.24)]{Lang2017} is identical to \eqref{eq:adi:ale1step}.
\end{remark}

\todo[inline]{%
Can $\rho(M) \leq \norm{M}$ (for any operator norm) be used to further simplify \cite[Algorithm~2.2, line~2]{Lang2017},
\ie can $\norm{}_2$ be replaced by \eg $\norm{}_F$?
As we are dealing with low-rank formulations of known rank $r$,
given that we ensure full rank of the factors,
we can \enquote{safely} exploit $\norm{A}_2 \leq \norm{A}_F \leq \sqrt{r}\norm{A}_2$.
}

\section{Parameter Selection}
\label{sec:adi:parameters}

After deriving the structure of the algorithm for \ac{ALE},
and the need of $\beta_k := \conj{\alpha_k}$ for a low-rank formulation,
we have to determine suitable parameters $\Set{\alpha_k : k\in\N}$.
Classical choices of pre-computed values include
optimal Wachspress shifts \cite{Wachspress1992,Wachspress2013} and
heuristic Penzl shifts \cite{Penzl1999}.
We resort to self-generating shifts described in \cite[Section~5.3]{Kuerschner2016}.

\section{Alternative Lyapunov Solvers}
Krylov subspace,
projection-based,
hybrid
