\section{Motivation}

\begin{frame}{Motivation}
  \begin{columns}[t,onlytextwidth]
  \column{0.5\linewidth}
  \begin{block}{\strut Optimal Control Problem}
    Consider the \acf{LQR} problem
    \begin{equation*}
      \begin{array}{cl}
        \min_u & \int_{t_0}^{t_f} y^\T y + u^\T u \dt + \tfrac{1}{100} y(t_f)^\T y(t_f) \medskip\\
        \text{s.t.} & \begin{aligned}[t]
          E \dot x &= Ax + Bu \\
          y &= Cx
        \end{aligned}
      \end{array}
    \end{equation*}
    using
    \begin{itemize}
      \item
        state $x(t)\in\R^n$
      \item
        control $u(t)\in\R^m$, $m\ll n$
      \item
        output $y(t)\in\R^q$, $q\ll n$
      \item
        constant system matrices $E,A,B,C$.
    \end{itemize}
  \end{block}
  \pause
  \column{0.47\linewidth}
  \begin{block}{\strut Feedback Law \parencite[e.g.][]{Locatelli2011}}
    The optimal control is given by
    \begin{equation*}
      u(t) = -
        B^\T \structure X(t_f+t_0-t) E
      x(t)
    \end{equation*}
    where $\structure X(t)\in\R^{n\times n}$ solves the \acf{DRE}
    \begin{equation*}
      \left\{
        \begin{lgathered}
          E^\T \dot{\structure X} E = \begin{multlined}[t]
            C^\T C + A^\T \structure X E + E^\T \structure X A \\ - E^\T \structure X BB^\T \structure X E
          \end{multlined} \\
          E^\T \structure X(t_0) E = \tfrac{1}{100} C^\T C
          .
        \end{lgathered}
      \right.
    \end{equation*}
  \end{block}
  \end{columns}
\end{frame}

\begin{frame}{Oberwolfach Rail}
\begin{columns}[onlytextwidth]
\column{0.6\textwidth}
  \begin{itemize}
    \item
      \cite{Benner2005}
    \item
      Linearized 2D heat equation:
      \begin{align*}
        c \rho\, \partial_t x &= \lambda\,\Delta x, && x\in\Omega\\
        \lambda\, \partial_n x &= q_k (u_k - x), && x \in\Gamma_k, k=1,\ldots,7\\
        \partial_n x &= 0, && x\in\Gamma_0
      \end{align*}
    \item
      FEM discretizations / mesh refinements with
      $n\in\Set{371,1357,5177,\num{20209},\ldots}$ states,\\
      $m=7$ inputs,
      $q=6$ outputs
    \item
      Time horizon $t\in[\SI{0}{\second}, \SI{45}{\second}]$
    \item
      Resulting $(E, A, B, C)$ system:
      $E \succ 0$ and $A \prec 0$\\
      $\leadsto$ uniquely solvable
  \end{itemize}
\column{0.37\textwidth}
  \includegraphics[width=\textwidth]{figures/Steelprofile1.jpg}
\end{columns}
\end{frame}

\begin{frame}<1>{Motivation}
\begin{bigpicturecols}
  \begin{itemize}
    \item
      Large-scale problems: a single (dense) $X(t)\in\Rnn$\\
      takes \SI{200}{\tera\byte} for
      $n=\num{5000000}$
    \item
      Solution $X$ is symmetric and dense, but has low rank

      \parencite[e.g.][]{Penzl2000}
    \item[$\leadsto$]
      Approximation via \rlap{\acl{LRSIF}:}
      \begin{equation*}
        \bigmat{X} \approx \mathop{\tallmat{L}} \mathop{\smallmat{D}} \mathop{\widemat{L^{\smash{\T}}}}
      \end{equation*}
    \item
      DRE is highly stiff, \ie\\
      high accuracy requires small time steps or high orders
    \item[$\leadsto$]
      Parareal method for speed-up
  \end{itemize}
\column{\bigpicturewidth}
\bigpicture{1}
\end{bigpicturecols}
\end{frame}
