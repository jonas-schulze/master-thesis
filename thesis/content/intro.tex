\chapter{Introduction}

Current challenges in science and engineering require a fast iteration on (intermediate) ideas and designs,
which in turn demands highly performant algorithms.
In this context, efficiency may only be a secondary concern.
Meanwhile, modern computer architectures are dominated by parallelism:
multiple cores per CPU,
multiple CPUs per node,
multiple nodes per cluster.
Many real-world application cans be described by means of \acp{PDE} or \acp{ODE},
which describe the behavior of an underlying system in space and time.
A common strategy is to solve these equations \emph{parallel-in-space},
\ie to distribute the spatial domain onto several CPUs,
which are then usually worked on in lockstep along the time dimension.
This strategy only scales as far as the local problems don't become too small,
due to the communication necessary along the interfaces of spatial sub-domains.
Another strategy is to (also) solve the problem \emph{parallel-in-time},
%\ie to solve the temporal dynamics of the system in parallel.
\ie to distribute the temporal domain onto several CPUs.
One algorithm of the latter strategy is the so-called parareal method~\cite{Lions2001},
which is the focus of this thesis.
%From a pure energy-consumed or efficiency perspective,
From a pure efficiency perspective in terms of energy consumed,
the parareal method is not worth pursuing.
However, as engineering demands fast cycle times,
this method can lead to large speed-ups.

In this thesis, above method shall be applied to the matrix-valued \ac{DRE},
which arises \eg in optimal control and model order reduction.
Practical applications quickly lead to large-scale systems,
for which the solution matrices easily reach thousands of rows and columns.
In order to handle them efficiently,
one uses the problem's inherent rapid decay of singular values
to represent the solution matrices in a low-rank factorization, see \eg~\cite{Penzl2000,Kuerschner2016,Lang2017}.
As the rank of the solution is not constant for all the interfaces of the temporal sub-domains,
implementation is somewhat difficult in well-established frameworks like OpenMPI\footnote{%
  \url{https://www.open-mpi.org/}
}.
Therefore, since Julia~\cite{Julia} promises to provide a fresh perspective on the problems of high-performance computing,
this thesis serves as a test project for just that.

The thesis is structured as follows.
First, \autoref{sec:HJT} gives an overview on the \ac{DRE} arising in optimal control of a linear dynamical system,
and \autoref{sec:matrixeq} presents the general techniques used to handle large-scale problems.
Then, \autoref{sec:ros} shows how to adapt a first-order and a second-order Rosenbrock method to the matrix-valued \ac{DRE}.
Every step of the Rosenbrock method requires the solution of an \ac{ALE},
which may be done using \eg the \ac{ADI} method described in \autoref{sec:ADI}.
This completes the general building blocks of a \ac{DRE} solver.
To speed-up this solver,
\autoref{sec:pr} shows how to apply the parareal method to the particular low-rank factorization used in this thesis.
\autoref{sec:impl} describes the implementation of aforementioned algorithms in the form of two Julia packages
as well as the numerical results of applying sequential and parareal solvers to
a small configuration of the Steel Profile benchmark problem~\cite{morwiki_steel},
comparing low-rank and dense versions of all algorithms.
Finally, \autoref{sec:conclusion} briefly summarizes findings and future research or optimization opportunities.

\section*{Hardware and Software Used}

The numerical experiments throughout this thesis were performed
on up to 29 standard nodes (450 cores) of the linux cluster Mechthild\footnote{\url{https://www.mpi-magdeburg.mpg.de/cluster/mechthild}}
at the Max Planck Institute for Dynamics of Complex Technical Systems in Magdeburg,
each having
two Intel Xeon Skylake Silver 4110 with 8 cores per CPU
and \SI{192}{\giga\byte} of memory,
or on a laptop (MacBook Pro, 13-inch, 2018, macOS 11) having
Intel Core i5-8259U with 4 cores and \SI{16}{\giga\byte} of memory.
Unless stated otherwise, each process uses a single thread.

\julia{DrWatson.jl}~\cite{DrWatson} was used to assist data management.
The dense \ac{DRE} solvers are built on top of \julia{MatrixEquations.jl}~\cite{MatrixEquations}.
\julia{DifferentialEquations.jl}~\cite{DifferentialEquations} was used for \autoref{example:parareal}.
The figures in this thesis were created with
\julia{Makie.jl}~\cite{Makie} or Ti\emph{k}Z~\cite{TikZ}.
Julia was used in version 1.6.1 on Mechthild and 1.6.2 on the laptop.

\section*{Sources to this Document}

The \LaTeX{} and Julia source codes for this thesis may be found at:
\begin{center}
\begin{tabular}{r}
  \url{https://gitlab.mpi-magdeburg.mpg.de/jschulze/master-thesis} \\
  \href{https://gitlab.mpi-magdeburg.mpg.de/jschulze/DifferentialRiccatiEquations.jl}{\ttfamily .../DifferentialRiccatiEquations.jl} \\
  \href{https://gitlab.mpi-magdeburg.mpg.de/jschulze/ParaReal.jl}{\ttfamily .../ParaReal.jl}
\end{tabular}
\end{center}
The present version of the document corresponds to git hash
\code{\input{githash.txt}}\unskip.
