\documentclass[
  aspectratio=1610,
]{beamer}

\includeonlyframes{rail1357,current,other}

\usepackage[american]{babel}
\usepackage[utf8]{inputenc}
\usepackage[T1]{fontenc}

% needs https://tex.stackexchange.com/questions/423848/xelatex-xy-and-dejavu-otf#423854
%\usepackage{dejavu-otf} % default Makie font: DejaVu Sans
%\usefonttheme{professionalfonts}

\title{A Low-Rank Parareal Solver for\\ Differential Riccati Equations\\ Written in Julia}
\author{Jonas Schulze}
\institute{Faculty of Mathematics\\ Otto-von-Guericke-Universität Magdeburg}
\date{May 10, 2022}
\subject{subject}

% beamer appearance
\setbeamercolor{block body}{bg=mpi} % for debugging
\setbeamercovered{transparent}
\beamertemplatenavigationsymbolsempty

\newcommand\maketocframe[1][]{%
  \begin{frame}{Outline}
    \tableofcontents[#1]
  \end{frame}
}

\AtBeginSection{%
  \maketocframe[currentsection,currentsubsection]
}

\usepackage[
  style=authoryear,
]{biblatex}
\addbibresource{stuff.bib}

\usepackage{mathtools}
\usepackage{xparse,xspace}
\usepackage[shortcuts]{glossaries}
\usepackage[binary-units]{siunitx}

\usepackage{tikz}
\usetikzlibrary{positioning}

\newcommand{\N}{\mathbb{N}} % Natural numbers
\newcommand{\Z}{\mathbb{Z}} % Whole numbers
\newcommand{\Q}{\mathbb{Q}} % Rational numbers
\newcommand{\R}{\mathbb{R}} % Real numbers
\newcommand{\C}{\mathbb{C}} % Complex numbers
\newcommand{\K}{\mathbb{K}} % Arbitrary Field

% transpose and conjugate/Hermitian transpose:
\newcommand\T{T}
\newcommand\HT{H}

% usage: \{2x\given x\in\N}
\newcommand{\given}{\mid}

% usage: \Set[\big]{2x \given x\in\N}
\newcommand\SetSymbol[1][]{%
  \nonscript\:#1\vert
  \allowbreak
  \nonscript\:
  \mathopen{}}
\DeclarePairedDelimiterX{\Set}[1]{\lbrace}{\rbrace}{%
  \renewcommand\given{\SetSymbol[\delimsize]}% this effect is local only
  #1%
}

% personal taste:
\let\epsilon\varepsilon
\renewcommand{\to}{\longrightarrow}
\renewcommand{\mapsto}{\longmapsto}
\renewcommand{\gets}{\longleftarrow}

% some more delimiters:
\NewDocumentCommand{\optional}{m}{\ifblank{#1}{\,\cdot\,}{#1}}
\DeclarePairedDelimiterX{\abs}[1]{\lvert}{\rvert}{\optional{#1}}
\DeclarePairedDelimiterX{\norm}[1]{\lVert}{\rVert}{\optional{#1}}
\DeclarePairedDelimiterX{\scalar}[2]{\langle}{\rangle}{\optional{#1},\optional{#2}}
\newcommand{\card}{\abs}

% integration:
\NewDocumentCommand{\intd}{m}{\,\textup{d}#1}
\newcommand\dt{\intd{t}}

\newcommand\julia\texttt
\newcommand\code\texttt

% https://tex.stackexchange.com/questions/22561/what-is-the-proper-use-of-i-e-backslash-at?noredirect=1&lq=1
\makeatletter % no idea why this is needed. \@ifnextchar doesn't work without it.
\newcommand\cf{cf.\@\xspace} % confer
\newcommand\eg{e.g.\@\xspace} % exempli gratia
\newcommand\etc{etc\@ifnextchar.{}{.\@\xspace}} % et cetera
\newcommand\ie{i.e.\@\xspace} % id est
\newcommand\wrt{w.r.t.\@\xspace} % with respect to
\makeatother

\definecolor{mathcore} {RGB}{102,  99, 100}
\definecolor{ovgu math}{RGB}{209,  63,  88}
\definecolor{mpi}      {RGB}{ 61, 167, 197}
\def\cola{ovgu math}
\def\colb{mathcore}
\def\colc{mpi}

\tikzset{
  mat/.style={
    rectangle,
    minimum size=1ex,
    inner sep=0mm,
  },
  bigmat/.style={mat,minimum size=#1},
  bigmat/.default=1cm,
  tallmat/.style={mat,minimum height=#1},
  tallmat/.default=1cm,
  widemat/.style={mat,minimum width=#1},
  widemat/.default=1cm,
  smallmat/.style={mat},
}

\newcommand\mat[2]{%
  \tikz[baseline=(M.base)] \node [mat, #1, fill=ovgu math] (M) {$#2$};
}
\newcommand\bigmat[1]{\mat{minimum size=2cm}{#1}}
\newcommand\tallmat[2][6mm]{\mat{minimum size=#1, minimum height=2cm}{#2}}
\newcommand\widemat[2][6mm]{\mat{minimum size=#1, minimum width=2cm}{#2}}
\newcommand\smallmat[2][6mm]{\mat{minimum size=#1}{#2}}


\renewcommand\mathrm\mathsf % fix \umach

\begin{document}

\frame[plain]{\titlepage}
\maketocframe

\section{Motivation}

\everymath{\displaystyle}

\begin{frame}{Motivation}
  \begin{columns}[t,onlytextwidth]
  \column{0.5\linewidth}
  \begin{block}{\strut Optimal Control Problem}
    Consider the \ac{LQR} problem
    \begin{equation*}
      \begin{array}{cl}
        \min_u & \int_{t_0}^{t_f} y^\T y + u^\T u \dt + \tfrac{1}{100} y(t_f)^\T y(t_f) \medskip\\
        \text{s.t.} & \begin{aligned}[t]
          E \dot x &= Ax + Bu \\
          y &= Cx
        \end{aligned}
      \end{array}
    \end{equation*}
    using
    \begin{itemize}
      \item
        state $x(t)\in\R^n$
      \item
        control $u(t)\in\R^m$, $m\ll n$
      \item
        output $y(t)\in\R^q$, $q\ll n$
      \item
        autonomous system matrices $E, A \in\Rnn$, $B\in\R^{n\times m}$, $C\in\R^{q\times n}$
    \end{itemize}
  \end{block}
  \column{0.45\linewidth}
  \begin{block}{\strut Feedback Law}
    The optimal control is given by
    \begin{equation*}
      u(t) = - \underbrace{
        B^\T X(t) E
      }_{
        K\mathrlap{(t)}
      }
      x(t)
    \end{equation*}
    where $X(t)\in\R^{n\times n}$ solves the \ac{DRE}
    \begin{equation*}
      \left\{
        \begin{aligned}
          E\dot X E &= C^\T C + A^\T X E + E^\T X A - E^\T X BB^\T X E \\
          E^\T X(t_0) E &= \tfrac{1}{100} C^\T C
        \end{aligned}
      \right.
    \end{equation*}
  \end{block}
  \end{columns}
\end{frame}

\begin{frame}{Notes}
  \begin{itemize}
    \item
      \enquote{low-rank} versions of a matrix or algorithm refer to \enquote{LRSIF}
  \end{itemize}
\end{frame}

\begin{frame}
  \begin{itemize}
    \item
      For a resolution of \SI{1}{\milli\second} along $[t_0,t_f] = [0, \SI{45}{\second}]$,
      storing $X$ takes ... memory

      $\leadsto$ only store $K := B^\T X E$
    \item
      For moderate size $n=\num{20000}$, a single (dense) $X(t)\in\Rnn$ takes \SI{3.2}{\giga\byte}.

      But: solution usually has low numerical rank \\
      \parencite[e.g.][]{Lang2017,Kuerschner2016,Penzl2000}

      $\leadsto$ \ac{LRSIF}
      \begin{equation}
        \bigmat{X} = \mathop{\tallmat{L}} \mathop{\smallmat{D}} \mathop{\widemat{L^{\smash{\T}}}}
      \end{equation}
    \item
      For a resolution of \SI{1}{\milli\second} along $[t_0,t_f] = [0, \SI{45}{\second}]$,
      and small $n=371$,
      computing a 4th order trajectory sequentially takes about 1 day.

      $\leadsto$ need further parallelization
  \end{itemize}
\end{frame}

\newcommand\tallcmat[1]{\tikz[baseline=-0.5ex]\node[tallmat,fill=#1] {};}
\newcommand\smallcmat[1]{\tikz\node[smallmat,fill=#1] {};}
\newcommand\widecmat[1]{\tikz\node[widemat,fill=#1] {};}
\newcommand\colorldlt[1]{%
  \mathop{\tallcmat{#1}}
  \mathop{\smallcmat{#1}}
  \mathop{\widecmat{#1}}
}

\newcommand\colorspacing{%
  \arraycolsep=3pt
  \def\arraystretch{0.75}
}

\begin{frame}{Low-Rank Symmetric Indefinite Factorization}
  \begin{itemize}
    \item \cite{Benner2009}
    \item Addition/Subtraction:
      \begin{equation*}
        \colorspacing
        \colorldlt{\cola} \pm \colorldlt{\colc}
        :=
        \Bigg[
        \begin{matrix}
          \tallcmat{\cola} &
          \tallcmat{\colc}
        \end{matrix}
        \Bigg]
        \begin{bmatrix}
          \smallcmat{\cola} \\
          & \pm \smallcmat{\colc}
        \end{bmatrix}
        % flag of the Netherlands:
        \begin{bmatrix}
          \widecmat{\cola} \\
          \widecmat{\colc}
        \end{bmatrix}
      \end{equation*}
    \item Problem: growing rank
      \begin{equation*}
        \colorspacing
        \colorldlt{\cola} + \colorldlt{\cola}
        :=
        \Bigg[
        \begin{matrix}
          \tallcmat{\cola} &
          \tallcmat{\cola}
        \end{matrix}
        \Bigg]
        \begin{bmatrix}
          \smallcmat{\cola} \\
          & \smallcmat{\cola}
        \end{bmatrix}
        % flag of Austria:
        \begin{bmatrix}
          \widecmat{\cola} \\
          \widecmat{\cola}
        \end{bmatrix}
        %\not\equiv
        %\mathop{\tallcmat{\cola}}
        %\mathop{(\smallcmat{\cola} + \smallcmat{\cola})}
        %\mathop{\widecmat{\cola}}
      \end{equation*}
      $\leadsto$ Column compression techniques
  \end{itemize}
\end{frame}

\subsection{Rosenbrock Method}

\begin{frame}<1>{Rosenbrock Method}
\begin{columns}
\column{0.7\textwidth}
  \begin{block}{General Formulation}
    For the initial value problem (IVP) $\dot x = f(x)$ the method reads
    \begin{equation*}
    \left\{
    \begin{aligned}
      x_{n+1} &:= x_n + \tau \sum_{j=1}^s b_j k_j
      \\
      k_i &:= \begin{multlined}[t]
      f\left( x_n + \tau \sum_{j=1}^{i-1} \alpha_{ij} k_j \right) + \tau \Jac \sum_{j=1}^i \gamma_{ij} k_j
      \\
      \text{for } i = 1, \ldots, s
      ,
      \end{multlined}
    \end{aligned}
    \right.
    \end{equation*}
    where $\Jac := f'(x_n)$ denotes the Jacobian.
  \end{block}
\column{0.3\textwidth}
\bigpicture{3}
\end{columns}
\end{frame}

\begin{frame}<1>{Rosenbrock Method}
\begin{columns}
\column{0.7\textwidth}
  \begin{block}{1st Order Scheme (Linearly Implicit Euler Scheme)}
    For the IVP $\dot x = f(x)$ the method reads
    \begin{equation*}
    \left\{
    \begin{aligned}
      x_{n+1} &= x_n + \tau k_1 \\
      (I - \gamma\tau \Jac) k_1 &= f(x_n)
      .
    \end{aligned}
    \right.
    \end{equation*}
  \end{block}
  \begin{block}{2nd Order Scheme \parencite{Verwer1999}}
    %\vspace*{-\baselineskip}
    %For the IVP $\dot x = f(x)$ the method reads
    \begin{equation*}
    \left\{
    \begin{aligned}
      x_{n+1} &= x_n + \tfrac{3}{2} \tau k_1 + \tfrac{1}{2} \tau k_2 \\
      (I - \gamma\tau \Jac) k_1 &= f(x_n) \\
      (I - \gamma\tau \Jac) k_2 &= f(x_n + \tau k_1) - 2k_1
    \end{aligned}
    \right.
    \end{equation*}
  \end{block}
\column{0.3\textwidth}
\bigpicture{3}
\end{columns}
\end{frame}

\begin{frame}<-4>{Rosenbrock Method}
\begin{columns}
\column{0.7\textwidth}
\begin{itemize}
\item
  Consider the DRE $E^\T \dot X E = \Ricc(X)$.
  \newcommand\U{\alert{\makebox[\widthof{$X_n$}]{$U$}}}
  \begin{align*}
    \Ricc(X_n) &= C^\T C + A^\T X_n E + E^\T X_n A - E^\T X_n BB^\T X_n E
    \\
    \pause
    \Jac(\alert{U}) &= \makebox[\widthof{$C^\T C$}]{$0$}
      + A^\T \U E + E^\T \U A
      \begin{lgathered}[t]
        {} - E^\T \U BB^\T X_n E \\
        {} - E^\T X_n BB^\T \U E
      \end{lgathered} \\
    \pause
    &= (A - BB^\T X_n E)^\T \alert U E + E^\T \alert U (A - BB^\T X_n E)
  \end{align*}
\item
  The Jacobian is a Lyapunov operator!
\pause
\item %TODO: when \leadsto, when \implies?
  $I - \gamma\tau\Jac$ is a Lyapunov operator,
  \ie all Rosenbrock stages are Algebraic Lyapunov Equations (ALEs).
\end{itemize}
\column{0.3\textwidth}
\bigpicture[3]{3}
\end{columns}
\end{frame}

\begin{frame}<1>{1st Order Rosenbrock Scheme}
\begin{columns}
\column{0.7\textwidth}
  \begin{itemize}
    \item
      Linearly implicit Euler scheme
    \item
      \cite{Mena2007}: Formulation for DRE $E^\T \dot X E = \Ricc(X)$
    \item
      \cite{Lang2017}: Formulation for LRSIF $X(t) = LDL^\T$
    \item
      1 ALE per step:
      \begin{equation*}
        \tilde A_n^\T X_{n+1} E + E^\T X_{n+1} \tilde A_n = - GSG^\T
      \end{equation*}
      where
      \begin{align*}
        \tilde A_n &= \gamma\tau(A-BB^\T X_n E) - \tfrac{1}{2} E
        \\
        G &= \begin{bmatrix}
          C^\T & E^\T L
        \end{bmatrix}
        \\
        S &= \begin{bmatrix}
          I & . \\
          . & DL^\T BB^\T LD + \tfrac{1}{\tau} D
        \end{bmatrix}
      \end{align*}
  \end{itemize}
  \vfill
\column{0.3\textwidth}
\bigpicture{4}
\end{columns}
\end{frame}

\begin{frame}<1>{2nd Order Rosenbrock Scheme}
\begin{columns}
\column{0.7\textwidth}
  \begin{itemize}
    \item
      \cite{Verwer1999}
    \item
      \cite{Mena2007}: Formulation for DRE $E^\T \dot X E = \Ricc(X)$
    \item
      \cite{Lang2017}: Formulation for LRSIF $X(t) = LDL^\T$
    \item
      2 ALEs per step:
      \begin{equation*}
      \left\{
      \begin{aligned}
        X_{n+1} &= X_n + \big( 2 - \tfrac{1}{2\gamma} \big) \tau K_1 - \tfrac{1}{2} \tau K_{21} \\
        \hat A_n^\T K_1 E + E^\T K_1 \hat A_n &= -\Ricc(X_n) \\
        \hat A_n^\T K_{21} E + E^\T K_{21} \hat A_n &= -\big( \tau^2 K_1 BB^\T K_1 + \big( 2-\tfrac{1}{\gamma} \big) K_1 \big)
      \end{aligned}
      \right.
      \end{equation*}
      (LRSIF right-hand sides not shown)
%    \item
%      Embedded 1st order method: $\tilde X_{n+1} = X_n + \gamma\tau K_1$
  \end{itemize}
\column{0.3\textwidth}
\bigpicture{4}
\end{columns}
\end{frame}

\subsection{Parareal Method}

\begin{frame}<-2>{Parareal Method}
\begin{bigpicturecols}
  \begin{block}{General Formulation \parencite{Lions2001}}
    For the IVP $\dot u = f(u)$ the method reads
    \begin{equation*}
      \left\{
      \begin{aligned}
        U^0_{n+1} &:= G(U^0_n) \\
        U^{k+1}_{n+1} &:= G(U^{k+1}_n) + F(U^k_n) - G(U^k_n)
      \end{aligned}
      \right.
    \end{equation*}
    where $U_0^0 := u(t_0)$. $U_n^k$ converges to $u(t_n)$ as $k\to\infty$.
  \end{block}
  \begin{itemize}
    \item
      Köhler, Saak, and Lang 2016 (GAMM):
      LRSIF formulation
  % GAMM Annual Meeting
  \begin{equation*}
    U^{k+1}_{n+1}
    =\vphantom{\Bigg[}\parbox{6cm}{$%
    \alt<1>{
      \colorldlt{\cola}
    + \colorldlt{\colb}
    - \colorldlt{\colc}
    }{
    \colorspacing % must be located in respective cell!
    \Bigg[
    \begin{matrix}
      \tallcmat{\cola} &
      \tallcmat{\colb} &
      \tallcmat{\colc}
    \end{matrix}
    \Bigg]
    \begin{bmatrix}
      \smallcmat{\cola} \\
      & \smallcmat{\colb} \\
      && -\smallcmat{\colc}
    \end{bmatrix}
    \begin{bmatrix}
      \widecmat{\cola} \\
      \widecmat{\colb} \\
      \widecmat{\colc}
    \end{bmatrix}
    }$}
  \end{equation*}
    \item
      Speed-up $\approx t_F/t_G$ for large $N$,
      where $0 \leq n \leq N$.
  \end{itemize}
\column{\bigpicturewidth}
\bigpicture{6}
\end{bigpicturecols}
\end{frame}

\begin{frame}<-18>[plain,label=parareal_anim]
  \setlength{\intextsep}{0pt}
  \only<-18>{%
    \setbeamertemplate{caption}[numbered]
    \setlength{\abovecaptionskip}{0pt}
    \renewcommand\thefigure{6.4} % number in thesis
  }
  \frametitle{Parareal Example}
\begin{columns}
\column{0.65\textwidth}
  \begin{figure}
  $U^{k+1}_{n+1} := G(U^{k+1}_n) + F(U^k_n) - G(U^k_n)$
  \foreach \k in {0,1} {%
  \foreach \n [evaluate={\i=int(1+\n+\k*7)}] in {0,...,6} {%
    \includegraphics<\i>[height=0.65\textwidth]{figures/parareal-anim/step-\k-\n.pdf}%
  }}%
  \includegraphics<15>[height=0.65\textwidth]{figures/parareal-anim/step-2-5.pdf}%
  \includegraphics<16>[height=0.65\textwidth]{figures/parareal-anim/step-2-6.pdf}%
  \includegraphics<17>[height=0.65\textwidth]{figures/parareal-anim/step-3-5.pdf}%
  \includegraphics<18>[height=0.65\textwidth]{figures/parareal-anim/step-3-6.pdf}%
  \includegraphics<19>[height=0.625\textwidth]{figures/slides_timeline_simple}% WTF is wrong with the height?!
  % The update is not always visible; for k=3 most take less than 0.001 seconds, while t_F is about
  \caption{\alt<-18>{%
    Parareal method applied to a linear ODE
  }{%
    Timeline diagram for $n \leq N=10$ and $k \leq K = 4$
  }}
  \end{figure}
\column{0.35\textwidth}
  %\definecolor{rainbow1}{rgb}{0.5019608f0, 0.0f0, 0.5019608f0}
%\definecolor{rainbow2}{rgb}{0.0f0, 0.0f0, 1.0f0}
%\definecolor{rainbow3}{rgb}{0.0f0, 0.5019608f0, 0.0f0}
%\definecolor{rainbow4}{rgb}{1.0f0, 0.64705884f0, 0.0f0}
%\definecolor{rainbow5}{rgb}{1.0f0, 0.0f0, 0.0f0}

\newcommand\pararealU[2]{\draw (2*#2*\xshift,0) +(\yangle:2*#1*\yshift) node (U#1#2) {$U_{#1}^{#2}$};}
\newcommand\pararealG[2]{\draw (2*#2*\xshift,0) +(\yangle:2*#1*\yshift+\yshift) node (G#1#2) {$G(U_{#1}^{#2})$};}
\newcommand\pararealF[2]{\draw (2*#2*\xshift+\xshift,0) +(\yangle:2*#1*\yshift+\yshift) node (F#1#2) {$F(U_{#1}^{#2})$};}

\begin{tikzpicture}[diag/.style={out=45,in=180}]
  \footnotesize
  \def\xshift{11mm}
  \def\yshift{8mm}
  \def\yangle{90}
  % initial value
  \action<+->{\pararealU{0}{0}}
  % k = 0, coarse solutions
  \foreach \n [evaluate={\nprev=int(\n-1)}] in {1,...,5} {%
  \action<+->{%
    \pararealG{\nprev}{0}
    \pararealU{\n}{0}
    \draw [->] (G\nprev0) -- (U\n0);
    \draw [->] (U\nprev0) -- (G\nprev0);
  }}
  % k = 0, fine solutions
  \action<+->{%
  \foreach \n in {0,...,4} {%
    \pararealF{\n}{0}
    \draw [->] (U\n0) -- (F\n0);
  }}
  % k = 0, transform to ghost
  \action<+->{}
  % k = 1, coarse solutions
  \action<+->{%
  \pararealU{1}{1}
  \draw [->] (F00) -- (U11);
  }
  \foreach \n [evaluate={\nprev=int(\n-1)}] in {2,...,5} {%
  \action<+->{%
    \pararealG{\nprev}{1}
    \pararealU{\n}{1}
    \draw [->] (G\nprev1) -- (U\n1);
    \draw [->] (F\nprev0) -- (U\n1);
    \draw [->] (G\nprev0) to [diag] (U\n1);
    \draw [->] (U\nprev1) -- (G\nprev1);
  }}
  % k = 1, fine solutions
  \action<+->{%
  \foreach \n in {1,...,4} {%
    \pararealF{\n}{1}
    \draw [->] (U\n1) -- (F\n1);
  }}
  % k = 2, coarse solutions
  %\action<+->{%
  %\pararealU{2}{2}
  %\draw [->] (F11) -- (U22);
  %\foreach \n [evaluate={\nprev=int(\n-1)}] in {3,...,5} {%
  %  \pararealG{\nprev}{2}
  %  \pararealU{\n}{2}
  %  \draw [->] (G\nprev2) -- (U\n2);
  %  \draw [->] (F\nprev1) -- (U\n2);
  %  \draw [->] (G\nprev1) to [diag] (U\n2);
  %  \draw [->] (U\nprev2) -- (G\nprev2);
  %}}
  % k = 2, fine solutions
  %\action<+->{%
  %\foreach \n in {2,...,4} {%
  %  \pararealF{\n}{2}
  %  \draw [->] (U\n2) -- (F\n2);
  %}}
  % rest
  \action<+->{%
  \foreach \n in {1,...,4} {%
    \draw (F\n1.north east)+(40:1ex) node [rotate=35] {$\cdots$};
  }}
\end{tikzpicture}

\end{columns}
\end{frame}

% Move comparison of dependencies and timeline to its own frame:
\againframe<19>[plain]{parareal_anim}

\section{Results}

\subsection{Numerical Results}

\begin{frame}{Numerical Results}
  \setbeamertemplate{caption}[numbered]
  \begin{columns}
  \column{0.7\textwidth}
  \begin{minipage}[b][0.75\textwidth][c]{\textwidth}
\only<+>{
  \begin{figure}
  \renewcommand\thefigure{7.7} % number in thesis
  \includegraphics[width=\textwidth]{figures/fig_results_parareal.pdf}%
  \caption{Trajectory $X_{1,77}$ and relative error in $K$ for Rail371}
  \end{figure}
}
\only<+>{
  \begin{figure}
  \renewcommand\thefigure{7.8} % number in thesis
  \includegraphics[width=\textwidth]{figures/fig_results_parareal_rank.pdf}%
  \caption{Rank of $X=LDL^\T$ for Rail371}
  \end{figure}
}
\only<+>{
  \begin{figure}
  \renewcommand\thefigure{7.9} % number in thesis
  \includegraphics[width=\textwidth]{figures/fig_timeline_all.pdf}%
  \caption{Timeline chart of parareal method applied to Rail371}
  \end{figure}
}
  \end{minipage}
  \column{0.3\textwidth}
  \begin{block}{Parareal Setup}
    \begin{itemize}
      \item
        450 coarse steps (\SI{100}{\milli\second})
      \item
        100 fine steps per coarse step (\SI{1}{\milli\second})
      \item
        Maximum \#iterations: 10
      \item
        Convergence:
        relative change in~$X$\\ below $371\umach$,\\
        twice in a row,\\
        and all previous stages converged
    \end{itemize}
  \end{block}
  \end{columns}
\end{frame}

\subsection{Parallel Scaling}

\begin{frame}[b,fragile,label=current]{Parallel Scaling}
  \setbeamertemplate{caption}[numbered]
  \begin{columns}[c]
  \column{0.65\textwidth}
  \begin{table}
  %TODO: use hanging captions
  \renewcommand\thetable{7.3} % number in thesis
  \caption{%
    Speed-up and parallel efficiency of parareal method applied to Rail371 using $N=450$ cores.
    (timings in seconds)
  }
  \begin{tabular}{%
    l
    S[table-format=4.2] % par
    S[table-format=6.2] % seq est
    S[table-format=2.2] % speedup
    S[round-precision=3, round-minimum=0.001, table-format=1.3, table-space-text-post=$^{*}$] % efficiency
  }
    \toprule
    Solver &
    {$\tpar$} &
    {$\hattseq$} &
    {$\frac{\hattseq}{\tpar}$} &
    {$\frac{\hattseq}{N\cdot\tpar}$} \\
    \midrule
    \input{tables/speedup450_lr.tex}
    \addlinespace
    Dense 1/1 & 3627.2994508743286 & 75037.70862150192 & 20.686935180776086 & 0.0459709670683913 \\
Dense 1/2 & 3956.8617849349976 & 84595.60539913177 & 21.379469386879656 & 0.04750993197084368 \\
Dense 2/2 & 3602.3622279167175 & 89274.93912768364 & 24.78233266933632 & 0.05507185037630293 \\

    \addlinespace
    \input{tables/speedup450_ref.tex}
    \midrule
    \pause
    Rail1357 & 3001.4058759212494 & 22684.368317604065 & 7.557914275969537 & 0.016795365057710083$^{*}$ \\
    \bottomrule
  \end{tabular}
  \end{table}
  \column{0.35\textwidth}
  \begin{block}{Addendum}
  \begin{itemize}
    \item
      Actual runtime of (sequential) Dense 4:

      $\tseq < \SI{86831}{\second}$

      (Slurm job duration)
    \item
      LRSIF 1/1 applied to Rail1357:
      % TODO: goto button for timeline (and back button there)

      \begin{itemize}
        \item
          2 BLAS threads\\ per process
        \item
          $2\times$ round-robin scheduling onto\\
          $P=225$ processes
        \item[{\makebox[\widthof{\usebeamertemplate{itemize item}}][c]{$\ast$}}]
          actual efficiency:
          $2\hattseq/2P\cdot \tpar = \num[round-precision=3]{0.033590730115420166}$
      \end{itemize}

      % $\tpar = \SI{3600}{\second}$ (Slurm job duration)
  \end{itemize}
  \end{block}
  \end{columns}
  \onslide
  \vfill
  \begin{lstlisting}
MY_KIND=dense MY_NF=100 MY_OF=1 MY_OC=1 sbatch -n450 -J de11 par.job
  \end{lstlisting}
\end{frame}


\section{Summary}

\begin{frame}
  \frametitle{Summary}
  foo
\end{frame}

\appendix

\documentclass[
  aspectratio=1610,
]{beamer}

\includeonlyframes{rail1357,current,other}

\usepackage[american]{babel}
\usepackage[utf8]{inputenc}
\usepackage[T1]{fontenc}

% needs https://tex.stackexchange.com/questions/423848/xelatex-xy-and-dejavu-otf#423854
%\usepackage{dejavu-otf} % default Makie font: DejaVu Sans
%\usefonttheme{professionalfonts}

\title{A Low-Rank Parareal Solver for\\ Differential Riccati Equations\\ Written in Julia}
\author{Jonas Schulze}
\institute{Faculty of Mathematics\\ Otto-von-Guericke-Universität Magdeburg}
\date{May 10, 2022}
\subject{subject}

% beamer appearance
\setbeamercolor{block body}{bg=mpi} % for debugging
\setbeamercovered{transparent}
\beamertemplatenavigationsymbolsempty

\newcommand\maketocframe[1][]{%
  \begin{frame}{Outline}
    \tableofcontents[#1]
  \end{frame}
}

\AtBeginSection{%
  \maketocframe[currentsection,currentsubsection]
}

\usepackage[
  style=authoryear,
]{biblatex}
\addbibresource{stuff.bib}

\usepackage{mathtools}
\usepackage{xparse,xspace}
\usepackage[shortcuts]{glossaries}
\usepackage[binary-units]{siunitx}

\usepackage{tikz}
\usetikzlibrary{positioning}

\newcommand{\N}{\mathbb{N}} % Natural numbers
\newcommand{\Z}{\mathbb{Z}} % Whole numbers
\newcommand{\Q}{\mathbb{Q}} % Rational numbers
\newcommand{\R}{\mathbb{R}} % Real numbers
\newcommand{\C}{\mathbb{C}} % Complex numbers
\newcommand{\K}{\mathbb{K}} % Arbitrary Field

% transpose and conjugate/Hermitian transpose:
\newcommand\T{T}
\newcommand\HT{H}

% usage: \{2x\given x\in\N}
\newcommand{\given}{\mid}

% usage: \Set[\big]{2x \given x\in\N}
\newcommand\SetSymbol[1][]{%
  \nonscript\:#1\vert
  \allowbreak
  \nonscript\:
  \mathopen{}}
\DeclarePairedDelimiterX{\Set}[1]{\lbrace}{\rbrace}{%
  \renewcommand\given{\SetSymbol[\delimsize]}% this effect is local only
  #1%
}

% personal taste:
\let\epsilon\varepsilon
\renewcommand{\to}{\longrightarrow}
\renewcommand{\mapsto}{\longmapsto}
\renewcommand{\gets}{\longleftarrow}

% some more delimiters:
\NewDocumentCommand{\optional}{m}{\ifblank{#1}{\,\cdot\,}{#1}}
\DeclarePairedDelimiterX{\abs}[1]{\lvert}{\rvert}{\optional{#1}}
\DeclarePairedDelimiterX{\norm}[1]{\lVert}{\rVert}{\optional{#1}}
\DeclarePairedDelimiterX{\scalar}[2]{\langle}{\rangle}{\optional{#1},\optional{#2}}
\newcommand{\card}{\abs}

% integration:
\NewDocumentCommand{\intd}{m}{\,\textup{d}#1}
\newcommand\dt{\intd{t}}

\newcommand\julia\texttt
\newcommand\code\texttt

% https://tex.stackexchange.com/questions/22561/what-is-the-proper-use-of-i-e-backslash-at?noredirect=1&lq=1
\makeatletter % no idea why this is needed. \@ifnextchar doesn't work without it.
\newcommand\cf{cf.\@\xspace} % confer
\newcommand\eg{e.g.\@\xspace} % exempli gratia
\newcommand\etc{etc\@ifnextchar.{}{.\@\xspace}} % et cetera
\newcommand\ie{i.e.\@\xspace} % id est
\newcommand\wrt{w.r.t.\@\xspace} % with respect to
\makeatother

\definecolor{mathcore} {RGB}{102,  99, 100}
\definecolor{ovgu math}{RGB}{209,  63,  88}
\definecolor{mpi}      {RGB}{ 61, 167, 197}
\def\cola{ovgu math}
\def\colb{mathcore}
\def\colc{mpi}

\tikzset{
  mat/.style={
    rectangle,
    minimum size=1ex,
    inner sep=0mm,
  },
  bigmat/.style={mat,minimum size=#1},
  bigmat/.default=1cm,
  tallmat/.style={mat,minimum height=#1},
  tallmat/.default=1cm,
  widemat/.style={mat,minimum width=#1},
  widemat/.default=1cm,
  smallmat/.style={mat},
}

\newcommand\mat[2]{%
  \tikz[baseline=(M.base)] \node [mat, #1, fill=ovgu math] (M) {$#2$};
}
\newcommand\bigmat[1]{\mat{minimum size=2cm}{#1}}
\newcommand\tallmat[2][6mm]{\mat{minimum size=#1, minimum height=2cm}{#2}}
\newcommand\widemat[2][6mm]{\mat{minimum size=#1, minimum width=2cm}{#2}}
\newcommand\smallmat[2][6mm]{\mat{minimum size=#1}{#2}}


\renewcommand\mathrm\mathsf % fix \umach

\begin{document}

\frame[plain]{\titlepage}
\maketocframe

\section{Motivation}

\everymath{\displaystyle}

\begin{frame}{Motivation}
  \begin{columns}[t,onlytextwidth]
  \column{0.5\linewidth}
  \begin{block}{\strut Optimal Control Problem}
    Consider the \ac{LQR} problem
    \begin{equation*}
      \begin{array}{cl}
        \min_u & \int_{t_0}^{t_f} y^\T y + u^\T u \dt + \tfrac{1}{100} y(t_f)^\T y(t_f) \medskip\\
        \text{s.t.} & \begin{aligned}[t]
          E \dot x &= Ax + Bu \\
          y &= Cx
        \end{aligned}
      \end{array}
    \end{equation*}
    using
    \begin{itemize}
      \item
        state $x(t)\in\R^n$
      \item
        control $u(t)\in\R^m$, $m\ll n$
      \item
        output $y(t)\in\R^q$, $q\ll n$
      \item
        autonomous system matrices $E, A \in\Rnn$, $B\in\R^{n\times m}$, $C\in\R^{q\times n}$
    \end{itemize}
  \end{block}
  \column{0.45\linewidth}
  \begin{block}{\strut Feedback Law}
    The optimal control is given by
    \begin{equation*}
      u(t) = - \underbrace{
        B^\T X(t) E
      }_{
        K\mathrlap{(t)}
      }
      x(t)
    \end{equation*}
    where $X(t)\in\R^{n\times n}$ solves the \ac{DRE}
    \begin{equation*}
      \left\{
        \begin{aligned}
          E\dot X E &= C^\T C + A^\T X E + E^\T X A - E^\T X BB^\T X E \\
          E^\T X(t_0) E &= \tfrac{1}{100} C^\T C
        \end{aligned}
      \right.
    \end{equation*}
  \end{block}
  \end{columns}
\end{frame}

\begin{frame}{Notes}
  \begin{itemize}
    \item
      \enquote{low-rank} versions of a matrix or algorithm refer to \enquote{LRSIF}
  \end{itemize}
\end{frame}

\begin{frame}
  \begin{itemize}
    \item
      For a resolution of \SI{1}{\milli\second} along $[t_0,t_f] = [0, \SI{45}{\second}]$,
      storing $X$ takes ... memory

      $\leadsto$ only store $K := B^\T X E$
    \item
      For moderate size $n=\num{20000}$, a single (dense) $X(t)\in\Rnn$ takes \SI{3.2}{\giga\byte}.

      But: solution usually has low numerical rank \\
      \parencite[e.g.][]{Lang2017,Kuerschner2016,Penzl2000}

      $\leadsto$ \ac{LRSIF}
      \begin{equation}
        \bigmat{X} = \mathop{\tallmat{L}} \mathop{\smallmat{D}} \mathop{\widemat{L^{\smash{\T}}}}
      \end{equation}
    \item
      For a resolution of \SI{1}{\milli\second} along $[t_0,t_f] = [0, \SI{45}{\second}]$,
      and small $n=371$,
      computing a 4th order trajectory sequentially takes about 1 day.

      $\leadsto$ need further parallelization
  \end{itemize}
\end{frame}

\newcommand\tallcmat[1]{\tikz[baseline=-0.5ex]\node[tallmat,fill=#1] {};}
\newcommand\smallcmat[1]{\tikz\node[smallmat,fill=#1] {};}
\newcommand\widecmat[1]{\tikz\node[widemat,fill=#1] {};}
\newcommand\colorldlt[1]{%
  \mathop{\tallcmat{#1}}
  \mathop{\smallcmat{#1}}
  \mathop{\widecmat{#1}}
}

\newcommand\colorspacing{%
  \arraycolsep=3pt
  \def\arraystretch{0.75}
}

\begin{frame}{Low-Rank Symmetric Indefinite Factorization}
  \begin{itemize}
    \item \cite{Benner2009}
    \item Addition/Subtraction:
      \begin{equation*}
        \colorspacing
        \colorldlt{\cola} \pm \colorldlt{\colc}
        :=
        \Bigg[
        \begin{matrix}
          \tallcmat{\cola} &
          \tallcmat{\colc}
        \end{matrix}
        \Bigg]
        \begin{bmatrix}
          \smallcmat{\cola} \\
          & \pm \smallcmat{\colc}
        \end{bmatrix}
        % flag of the Netherlands:
        \begin{bmatrix}
          \widecmat{\cola} \\
          \widecmat{\colc}
        \end{bmatrix}
      \end{equation*}
    \item Problem: growing rank
      \begin{equation*}
        \colorspacing
        \colorldlt{\cola} + \colorldlt{\cola}
        :=
        \Bigg[
        \begin{matrix}
          \tallcmat{\cola} &
          \tallcmat{\cola}
        \end{matrix}
        \Bigg]
        \begin{bmatrix}
          \smallcmat{\cola} \\
          & \smallcmat{\cola}
        \end{bmatrix}
        % flag of Austria:
        \begin{bmatrix}
          \widecmat{\cola} \\
          \widecmat{\cola}
        \end{bmatrix}
        %\not\equiv
        %\mathop{\tallcmat{\cola}}
        %\mathop{(\smallcmat{\cola} + \smallcmat{\cola})}
        %\mathop{\widecmat{\cola}}
      \end{equation*}
      $\leadsto$ Column compression techniques
  \end{itemize}
\end{frame}

\subsection{Rosenbrock Method}

\begin{frame}<1>{Rosenbrock Method}
\begin{columns}
\column{0.7\textwidth}
  \begin{block}{General Formulation}
    For the initial value problem (IVP) $\dot x = f(x)$ the method reads
    \begin{equation*}
    \left\{
    \begin{aligned}
      x_{n+1} &:= x_n + \tau \sum_{j=1}^s b_j k_j
      \\
      k_i &:= \begin{multlined}[t]
      f\left( x_n + \tau \sum_{j=1}^{i-1} \alpha_{ij} k_j \right) + \tau \Jac \sum_{j=1}^i \gamma_{ij} k_j
      \\
      \text{for } i = 1, \ldots, s
      ,
      \end{multlined}
    \end{aligned}
    \right.
    \end{equation*}
    where $\Jac := f'(x_n)$ denotes the Jacobian.
  \end{block}
\column{0.3\textwidth}
\bigpicture{3}
\end{columns}
\end{frame}

\begin{frame}<1>{Rosenbrock Method}
\begin{columns}
\column{0.7\textwidth}
  \begin{block}{1st Order Scheme (Linearly Implicit Euler Scheme)}
    For the IVP $\dot x = f(x)$ the method reads
    \begin{equation*}
    \left\{
    \begin{aligned}
      x_{n+1} &= x_n + \tau k_1 \\
      (I - \gamma\tau \Jac) k_1 &= f(x_n)
      .
    \end{aligned}
    \right.
    \end{equation*}
  \end{block}
  \begin{block}{2nd Order Scheme \parencite{Verwer1999}}
    %\vspace*{-\baselineskip}
    %For the IVP $\dot x = f(x)$ the method reads
    \begin{equation*}
    \left\{
    \begin{aligned}
      x_{n+1} &= x_n + \tfrac{3}{2} \tau k_1 + \tfrac{1}{2} \tau k_2 \\
      (I - \gamma\tau \Jac) k_1 &= f(x_n) \\
      (I - \gamma\tau \Jac) k_2 &= f(x_n + \tau k_1) - 2k_1
    \end{aligned}
    \right.
    \end{equation*}
  \end{block}
\column{0.3\textwidth}
\bigpicture{3}
\end{columns}
\end{frame}

\begin{frame}<-4>{Rosenbrock Method}
\begin{columns}
\column{0.7\textwidth}
\begin{itemize}
\item
  Consider the DRE $E^\T \dot X E = \Ricc(X)$.
  \newcommand\U{\alert{\makebox[\widthof{$X_n$}]{$U$}}}
  \begin{align*}
    \Ricc(X_n) &= C^\T C + A^\T X_n E + E^\T X_n A - E^\T X_n BB^\T X_n E
    \\
    \pause
    \Jac(\alert{U}) &= \makebox[\widthof{$C^\T C$}]{$0$}
      + A^\T \U E + E^\T \U A
      \begin{lgathered}[t]
        {} - E^\T \U BB^\T X_n E \\
        {} - E^\T X_n BB^\T \U E
      \end{lgathered} \\
    \pause
    &= (A - BB^\T X_n E)^\T \alert U E + E^\T \alert U (A - BB^\T X_n E)
  \end{align*}
\item
  The Jacobian is a Lyapunov operator!
\pause
\item %TODO: when \leadsto, when \implies?
  $I - \gamma\tau\Jac$ is a Lyapunov operator,
  \ie all Rosenbrock stages are Algebraic Lyapunov Equations (ALEs).
\end{itemize}
\column{0.3\textwidth}
\bigpicture[3]{3}
\end{columns}
\end{frame}

\begin{frame}<1>{1st Order Rosenbrock Scheme}
\begin{columns}
\column{0.7\textwidth}
  \begin{itemize}
    \item
      Linearly implicit Euler scheme
    \item
      \cite{Mena2007}: Formulation for DRE $E^\T \dot X E = \Ricc(X)$
    \item
      \cite{Lang2017}: Formulation for LRSIF $X(t) = LDL^\T$
    \item
      1 ALE per step:
      \begin{equation*}
        \tilde A_n^\T X_{n+1} E + E^\T X_{n+1} \tilde A_n = - GSG^\T
      \end{equation*}
      where
      \begin{align*}
        \tilde A_n &= \gamma\tau(A-BB^\T X_n E) - \tfrac{1}{2} E
        \\
        G &= \begin{bmatrix}
          C^\T & E^\T L
        \end{bmatrix}
        \\
        S &= \begin{bmatrix}
          I & . \\
          . & DL^\T BB^\T LD + \tfrac{1}{\tau} D
        \end{bmatrix}
      \end{align*}
  \end{itemize}
  \vfill
\column{0.3\textwidth}
\bigpicture{4}
\end{columns}
\end{frame}

\begin{frame}<1>{2nd Order Rosenbrock Scheme}
\begin{columns}
\column{0.7\textwidth}
  \begin{itemize}
    \item
      \cite{Verwer1999}
    \item
      \cite{Mena2007}: Formulation for DRE $E^\T \dot X E = \Ricc(X)$
    \item
      \cite{Lang2017}: Formulation for LRSIF $X(t) = LDL^\T$
    \item
      2 ALEs per step:
      \begin{equation*}
      \left\{
      \begin{aligned}
        X_{n+1} &= X_n + \big( 2 - \tfrac{1}{2\gamma} \big) \tau K_1 - \tfrac{1}{2} \tau K_{21} \\
        \hat A_n^\T K_1 E + E^\T K_1 \hat A_n &= -\Ricc(X_n) \\
        \hat A_n^\T K_{21} E + E^\T K_{21} \hat A_n &= -\big( \tau^2 K_1 BB^\T K_1 + \big( 2-\tfrac{1}{\gamma} \big) K_1 \big)
      \end{aligned}
      \right.
      \end{equation*}
      (LRSIF right-hand sides not shown)
%    \item
%      Embedded 1st order method: $\tilde X_{n+1} = X_n + \gamma\tau K_1$
  \end{itemize}
\column{0.3\textwidth}
\bigpicture{4}
\end{columns}
\end{frame}

\subsection{Parareal Method}

\begin{frame}<-2>{Parareal Method}
\begin{bigpicturecols}
  \begin{block}{General Formulation \parencite{Lions2001}}
    For the IVP $\dot u = f(u)$ the method reads
    \begin{equation*}
      \left\{
      \begin{aligned}
        U^0_{n+1} &:= G(U^0_n) \\
        U^{k+1}_{n+1} &:= G(U^{k+1}_n) + F(U^k_n) - G(U^k_n)
      \end{aligned}
      \right.
    \end{equation*}
    where $U_0^0 := u(t_0)$. $U_n^k$ converges to $u(t_n)$ as $k\to\infty$.
  \end{block}
  \begin{itemize}
    \item
      Köhler, Saak, and Lang 2016 (GAMM):
      LRSIF formulation
  % GAMM Annual Meeting
  \begin{equation*}
    U^{k+1}_{n+1}
    =\vphantom{\Bigg[}\parbox{6cm}{$%
    \alt<1>{
      \colorldlt{\cola}
    + \colorldlt{\colb}
    - \colorldlt{\colc}
    }{
    \colorspacing % must be located in respective cell!
    \Bigg[
    \begin{matrix}
      \tallcmat{\cola} &
      \tallcmat{\colb} &
      \tallcmat{\colc}
    \end{matrix}
    \Bigg]
    \begin{bmatrix}
      \smallcmat{\cola} \\
      & \smallcmat{\colb} \\
      && -\smallcmat{\colc}
    \end{bmatrix}
    \begin{bmatrix}
      \widecmat{\cola} \\
      \widecmat{\colb} \\
      \widecmat{\colc}
    \end{bmatrix}
    }$}
  \end{equation*}
    \item
      Speed-up $\approx t_F/t_G$ for large $N$,
      where $0 \leq n \leq N$.
  \end{itemize}
\column{\bigpicturewidth}
\bigpicture{6}
\end{bigpicturecols}
\end{frame}

\begin{frame}<-18>[plain,label=parareal_anim]
  \setlength{\intextsep}{0pt}
  \only<-18>{%
    \setbeamertemplate{caption}[numbered]
    \setlength{\abovecaptionskip}{0pt}
    \renewcommand\thefigure{6.4} % number in thesis
  }
  \frametitle{Parareal Example}
\begin{columns}
\column{0.65\textwidth}
  \begin{figure}
  $U^{k+1}_{n+1} := G(U^{k+1}_n) + F(U^k_n) - G(U^k_n)$
  \foreach \k in {0,1} {%
  \foreach \n [evaluate={\i=int(1+\n+\k*7)}] in {0,...,6} {%
    \includegraphics<\i>[height=0.65\textwidth]{figures/parareal-anim/step-\k-\n.pdf}%
  }}%
  \includegraphics<15>[height=0.65\textwidth]{figures/parareal-anim/step-2-5.pdf}%
  \includegraphics<16>[height=0.65\textwidth]{figures/parareal-anim/step-2-6.pdf}%
  \includegraphics<17>[height=0.65\textwidth]{figures/parareal-anim/step-3-5.pdf}%
  \includegraphics<18>[height=0.65\textwidth]{figures/parareal-anim/step-3-6.pdf}%
  \includegraphics<19>[height=0.625\textwidth]{figures/slides_timeline_simple}% WTF is wrong with the height?!
  % The update is not always visible; for k=3 most take less than 0.001 seconds, while t_F is about
  \caption{\alt<-18>{%
    Parareal method applied to a linear ODE
  }{%
    Timeline diagram for $n \leq N=10$ and $k \leq K = 4$
  }}
  \end{figure}
\column{0.35\textwidth}
  %\definecolor{rainbow1}{rgb}{0.5019608f0, 0.0f0, 0.5019608f0}
%\definecolor{rainbow2}{rgb}{0.0f0, 0.0f0, 1.0f0}
%\definecolor{rainbow3}{rgb}{0.0f0, 0.5019608f0, 0.0f0}
%\definecolor{rainbow4}{rgb}{1.0f0, 0.64705884f0, 0.0f0}
%\definecolor{rainbow5}{rgb}{1.0f0, 0.0f0, 0.0f0}

\newcommand\pararealU[2]{\draw (2*#2*\xshift,0) +(\yangle:2*#1*\yshift) node (U#1#2) {$U_{#1}^{#2}$};}
\newcommand\pararealG[2]{\draw (2*#2*\xshift,0) +(\yangle:2*#1*\yshift+\yshift) node (G#1#2) {$G(U_{#1}^{#2})$};}
\newcommand\pararealF[2]{\draw (2*#2*\xshift+\xshift,0) +(\yangle:2*#1*\yshift+\yshift) node (F#1#2) {$F(U_{#1}^{#2})$};}

\begin{tikzpicture}[diag/.style={out=45,in=180}]
  \footnotesize
  \def\xshift{11mm}
  \def\yshift{8mm}
  \def\yangle{90}
  % initial value
  \action<+->{\pararealU{0}{0}}
  % k = 0, coarse solutions
  \foreach \n [evaluate={\nprev=int(\n-1)}] in {1,...,5} {%
  \action<+->{%
    \pararealG{\nprev}{0}
    \pararealU{\n}{0}
    \draw [->] (G\nprev0) -- (U\n0);
    \draw [->] (U\nprev0) -- (G\nprev0);
  }}
  % k = 0, fine solutions
  \action<+->{%
  \foreach \n in {0,...,4} {%
    \pararealF{\n}{0}
    \draw [->] (U\n0) -- (F\n0);
  }}
  % k = 0, transform to ghost
  \action<+->{}
  % k = 1, coarse solutions
  \action<+->{%
  \pararealU{1}{1}
  \draw [->] (F00) -- (U11);
  }
  \foreach \n [evaluate={\nprev=int(\n-1)}] in {2,...,5} {%
  \action<+->{%
    \pararealG{\nprev}{1}
    \pararealU{\n}{1}
    \draw [->] (G\nprev1) -- (U\n1);
    \draw [->] (F\nprev0) -- (U\n1);
    \draw [->] (G\nprev0) to [diag] (U\n1);
    \draw [->] (U\nprev1) -- (G\nprev1);
  }}
  % k = 1, fine solutions
  \action<+->{%
  \foreach \n in {1,...,4} {%
    \pararealF{\n}{1}
    \draw [->] (U\n1) -- (F\n1);
  }}
  % k = 2, coarse solutions
  %\action<+->{%
  %\pararealU{2}{2}
  %\draw [->] (F11) -- (U22);
  %\foreach \n [evaluate={\nprev=int(\n-1)}] in {3,...,5} {%
  %  \pararealG{\nprev}{2}
  %  \pararealU{\n}{2}
  %  \draw [->] (G\nprev2) -- (U\n2);
  %  \draw [->] (F\nprev1) -- (U\n2);
  %  \draw [->] (G\nprev1) to [diag] (U\n2);
  %  \draw [->] (U\nprev2) -- (G\nprev2);
  %}}
  % k = 2, fine solutions
  %\action<+->{%
  %\foreach \n in {2,...,4} {%
  %  \pararealF{\n}{2}
  %  \draw [->] (U\n2) -- (F\n2);
  %}}
  % rest
  \action<+->{%
  \foreach \n in {1,...,4} {%
    \draw (F\n1.north east)+(40:1ex) node [rotate=35] {$\cdots$};
  }}
\end{tikzpicture}

\end{columns}
\end{frame}

% Move comparison of dependencies and timeline to its own frame:
\againframe<19>[plain]{parareal_anim}

\section{Results}

\subsection{Numerical Results}

\begin{frame}{Numerical Results}
  \setbeamertemplate{caption}[numbered]
  \begin{columns}
  \column{0.7\textwidth}
  \begin{minipage}[b][0.75\textwidth][c]{\textwidth}
\only<+>{
  \begin{figure}
  \renewcommand\thefigure{7.7} % number in thesis
  \includegraphics[width=\textwidth]{figures/fig_results_parareal.pdf}%
  \caption{Trajectory $X_{1,77}$ and relative error in $K$ for Rail371}
  \end{figure}
}
\only<+>{
  \begin{figure}
  \renewcommand\thefigure{7.8} % number in thesis
  \includegraphics[width=\textwidth]{figures/fig_results_parareal_rank.pdf}%
  \caption{Rank of $X=LDL^\T$ for Rail371}
  \end{figure}
}
\only<+>{
  \begin{figure}
  \renewcommand\thefigure{7.9} % number in thesis
  \includegraphics[width=\textwidth]{figures/fig_timeline_all.pdf}%
  \caption{Timeline chart of parareal method applied to Rail371}
  \end{figure}
}
  \end{minipage}
  \column{0.3\textwidth}
  \begin{block}{Parareal Setup}
    \begin{itemize}
      \item
        450 coarse steps (\SI{100}{\milli\second})
      \item
        100 fine steps per coarse step (\SI{1}{\milli\second})
      \item
        Maximum \#iterations: 10
      \item
        Convergence:
        relative change in~$X$\\ below $371\umach$,\\
        twice in a row,\\
        and all previous stages converged
    \end{itemize}
  \end{block}
  \end{columns}
\end{frame}

\subsection{Parallel Scaling}

\begin{frame}[b,fragile,label=current]{Parallel Scaling}
  \setbeamertemplate{caption}[numbered]
  \begin{columns}[c]
  \column{0.65\textwidth}
  \begin{table}
  %TODO: use hanging captions
  \renewcommand\thetable{7.3} % number in thesis
  \caption{%
    Speed-up and parallel efficiency of parareal method applied to Rail371 using $N=450$ cores.
    (timings in seconds)
  }
  \begin{tabular}{%
    l
    S[table-format=4.2] % par
    S[table-format=6.2] % seq est
    S[table-format=2.2] % speedup
    S[round-precision=3, round-minimum=0.001, table-format=1.3, table-space-text-post=$^{*}$] % efficiency
  }
    \toprule
    Solver &
    {$\tpar$} &
    {$\hattseq$} &
    {$\frac{\hattseq}{\tpar}$} &
    {$\frac{\hattseq}{N\cdot\tpar}$} \\
    \midrule
    \input{tables/speedup450_lr.tex}
    \addlinespace
    Dense 1/1 & 3627.2994508743286 & 75037.70862150192 & 20.686935180776086 & 0.0459709670683913 \\
Dense 1/2 & 3956.8617849349976 & 84595.60539913177 & 21.379469386879656 & 0.04750993197084368 \\
Dense 2/2 & 3602.3622279167175 & 89274.93912768364 & 24.78233266933632 & 0.05507185037630293 \\

    \addlinespace
    \input{tables/speedup450_ref.tex}
    \midrule
    \pause
    Rail1357 & 3001.4058759212494 & 22684.368317604065 & 7.557914275969537 & 0.016795365057710083$^{*}$ \\
    \bottomrule
  \end{tabular}
  \end{table}
  \column{0.35\textwidth}
  \begin{block}{Addendum}
  \begin{itemize}
    \item
      Actual runtime of (sequential) Dense 4:

      $\tseq < \SI{86831}{\second}$

      (Slurm job duration)
    \item
      LRSIF 1/1 applied to Rail1357:
      % TODO: goto button for timeline (and back button there)

      \begin{itemize}
        \item
          2 BLAS threads\\ per process
        \item
          $2\times$ round-robin scheduling onto\\
          $P=225$ processes
        \item[{\makebox[\widthof{\usebeamertemplate{itemize item}}][c]{$\ast$}}]
          actual efficiency:
          $2\hattseq/2P\cdot \tpar = \num[round-precision=3]{0.033590730115420166}$
      \end{itemize}

      % $\tpar = \SI{3600}{\second}$ (Slurm job duration)
  \end{itemize}
  \end{block}
  \end{columns}
  \onslide
  \vfill
  \begin{lstlisting}
MY_KIND=dense MY_NF=100 MY_OF=1 MY_OC=1 sbatch -n450 -J de11 par.job
  \end{lstlisting}
\end{frame}


\section{Summary}

\begin{frame}
  \frametitle{Summary}
  foo
\end{frame}

\appendix

\documentclass[
  aspectratio=1610,
]{beamer}

\includeonlyframes{rail1357,current,other}

\usepackage[american]{babel}
\usepackage[utf8]{inputenc}
\usepackage[T1]{fontenc}

% needs https://tex.stackexchange.com/questions/423848/xelatex-xy-and-dejavu-otf#423854
%\usepackage{dejavu-otf} % default Makie font: DejaVu Sans
%\usefonttheme{professionalfonts}

\title{A Low-Rank Parareal Solver for\\ Differential Riccati Equations\\ Written in Julia}
\author{Jonas Schulze}
\institute{Faculty of Mathematics\\ Otto-von-Guericke-Universität Magdeburg}
\date{May 10, 2022}
\subject{subject}

% beamer appearance
\setbeamercolor{block body}{bg=mpi} % for debugging
\setbeamercovered{transparent}
\beamertemplatenavigationsymbolsempty

\newcommand\maketocframe[1][]{%
  \begin{frame}{Outline}
    \tableofcontents[#1]
  \end{frame}
}

\AtBeginSection{%
  \maketocframe[currentsection,currentsubsection]
}

\usepackage[
  style=authoryear,
]{biblatex}
\addbibresource{stuff.bib}

\usepackage{mathtools}
\usepackage{xparse,xspace}
\usepackage[shortcuts]{glossaries}
\usepackage[binary-units]{siunitx}

\usepackage{tikz}
\usetikzlibrary{positioning}

\newcommand{\N}{\mathbb{N}} % Natural numbers
\newcommand{\Z}{\mathbb{Z}} % Whole numbers
\newcommand{\Q}{\mathbb{Q}} % Rational numbers
\newcommand{\R}{\mathbb{R}} % Real numbers
\newcommand{\C}{\mathbb{C}} % Complex numbers
\newcommand{\K}{\mathbb{K}} % Arbitrary Field

% transpose and conjugate/Hermitian transpose:
\newcommand\T{T}
\newcommand\HT{H}

% usage: \{2x\given x\in\N}
\newcommand{\given}{\mid}

% usage: \Set[\big]{2x \given x\in\N}
\newcommand\SetSymbol[1][]{%
  \nonscript\:#1\vert
  \allowbreak
  \nonscript\:
  \mathopen{}}
\DeclarePairedDelimiterX{\Set}[1]{\lbrace}{\rbrace}{%
  \renewcommand\given{\SetSymbol[\delimsize]}% this effect is local only
  #1%
}

% personal taste:
\let\epsilon\varepsilon
\renewcommand{\to}{\longrightarrow}
\renewcommand{\mapsto}{\longmapsto}
\renewcommand{\gets}{\longleftarrow}

% some more delimiters:
\NewDocumentCommand{\optional}{m}{\ifblank{#1}{\,\cdot\,}{#1}}
\DeclarePairedDelimiterX{\abs}[1]{\lvert}{\rvert}{\optional{#1}}
\DeclarePairedDelimiterX{\norm}[1]{\lVert}{\rVert}{\optional{#1}}
\DeclarePairedDelimiterX{\scalar}[2]{\langle}{\rangle}{\optional{#1},\optional{#2}}
\newcommand{\card}{\abs}

% integration:
\NewDocumentCommand{\intd}{m}{\,\textup{d}#1}
\newcommand\dt{\intd{t}}

\newcommand\julia\texttt
\newcommand\code\texttt

% https://tex.stackexchange.com/questions/22561/what-is-the-proper-use-of-i-e-backslash-at?noredirect=1&lq=1
\makeatletter % no idea why this is needed. \@ifnextchar doesn't work without it.
\newcommand\cf{cf.\@\xspace} % confer
\newcommand\eg{e.g.\@\xspace} % exempli gratia
\newcommand\etc{etc\@ifnextchar.{}{.\@\xspace}} % et cetera
\newcommand\ie{i.e.\@\xspace} % id est
\newcommand\wrt{w.r.t.\@\xspace} % with respect to
\makeatother

\definecolor{mathcore} {RGB}{102,  99, 100}
\definecolor{ovgu math}{RGB}{209,  63,  88}
\definecolor{mpi}      {RGB}{ 61, 167, 197}
\def\cola{ovgu math}
\def\colb{mathcore}
\def\colc{mpi}

\tikzset{
  mat/.style={
    rectangle,
    minimum size=1ex,
    inner sep=0mm,
  },
  bigmat/.style={mat,minimum size=#1},
  bigmat/.default=1cm,
  tallmat/.style={mat,minimum height=#1},
  tallmat/.default=1cm,
  widemat/.style={mat,minimum width=#1},
  widemat/.default=1cm,
  smallmat/.style={mat},
}

\newcommand\mat[2]{%
  \tikz[baseline=(M.base)] \node [mat, #1, fill=ovgu math] (M) {$#2$};
}
\newcommand\bigmat[1]{\mat{minimum size=2cm}{#1}}
\newcommand\tallmat[2][6mm]{\mat{minimum size=#1, minimum height=2cm}{#2}}
\newcommand\widemat[2][6mm]{\mat{minimum size=#1, minimum width=2cm}{#2}}
\newcommand\smallmat[2][6mm]{\mat{minimum size=#1}{#2}}


\renewcommand\mathrm\mathsf % fix \umach

\begin{document}

\frame[plain]{\titlepage}
\maketocframe

\section{Motivation}

\everymath{\displaystyle}

\begin{frame}{Motivation}
  \begin{columns}[t,onlytextwidth]
  \column{0.5\linewidth}
  \begin{block}{\strut Optimal Control Problem}
    Consider the \ac{LQR} problem
    \begin{equation*}
      \begin{array}{cl}
        \min_u & \int_{t_0}^{t_f} y^\T y + u^\T u \dt + \tfrac{1}{100} y(t_f)^\T y(t_f) \medskip\\
        \text{s.t.} & \begin{aligned}[t]
          E \dot x &= Ax + Bu \\
          y &= Cx
        \end{aligned}
      \end{array}
    \end{equation*}
    using
    \begin{itemize}
      \item
        state $x(t)\in\R^n$
      \item
        control $u(t)\in\R^m$, $m\ll n$
      \item
        output $y(t)\in\R^q$, $q\ll n$
      \item
        autonomous system matrices $E, A \in\Rnn$, $B\in\R^{n\times m}$, $C\in\R^{q\times n}$
    \end{itemize}
  \end{block}
  \column{0.45\linewidth}
  \begin{block}{\strut Feedback Law}
    The optimal control is given by
    \begin{equation*}
      u(t) = - \underbrace{
        B^\T X(t) E
      }_{
        K\mathrlap{(t)}
      }
      x(t)
    \end{equation*}
    where $X(t)\in\R^{n\times n}$ solves the \ac{DRE}
    \begin{equation*}
      \left\{
        \begin{aligned}
          E\dot X E &= C^\T C + A^\T X E + E^\T X A - E^\T X BB^\T X E \\
          E^\T X(t_0) E &= \tfrac{1}{100} C^\T C
        \end{aligned}
      \right.
    \end{equation*}
  \end{block}
  \end{columns}
\end{frame}

\begin{frame}{Notes}
  \begin{itemize}
    \item
      \enquote{low-rank} versions of a matrix or algorithm refer to \enquote{LRSIF}
  \end{itemize}
\end{frame}

\begin{frame}
  \begin{itemize}
    \item
      For a resolution of \SI{1}{\milli\second} along $[t_0,t_f] = [0, \SI{45}{\second}]$,
      storing $X$ takes ... memory

      $\leadsto$ only store $K := B^\T X E$
    \item
      For moderate size $n=\num{20000}$, a single (dense) $X(t)\in\Rnn$ takes \SI{3.2}{\giga\byte}.

      But: solution usually has low numerical rank \\
      \parencite[e.g.][]{Lang2017,Kuerschner2016,Penzl2000}

      $\leadsto$ \ac{LRSIF}
      \begin{equation}
        \bigmat{X} = \mathop{\tallmat{L}} \mathop{\smallmat{D}} \mathop{\widemat{L^{\smash{\T}}}}
      \end{equation}
    \item
      For a resolution of \SI{1}{\milli\second} along $[t_0,t_f] = [0, \SI{45}{\second}]$,
      and small $n=371$,
      computing a 4th order trajectory sequentially takes about 1 day.

      $\leadsto$ need further parallelization
  \end{itemize}
\end{frame}

\newcommand\tallcmat[1]{\tikz[baseline=-0.5ex]\node[tallmat,fill=#1] {};}
\newcommand\smallcmat[1]{\tikz\node[smallmat,fill=#1] {};}
\newcommand\widecmat[1]{\tikz\node[widemat,fill=#1] {};}
\newcommand\colorldlt[1]{%
  \mathop{\tallcmat{#1}}
  \mathop{\smallcmat{#1}}
  \mathop{\widecmat{#1}}
}

\newcommand\colorspacing{%
  \arraycolsep=3pt
  \def\arraystretch{0.75}
}

\begin{frame}{Low-Rank Symmetric Indefinite Factorization}
  \begin{itemize}
    \item \cite{Benner2009}
    \item Addition/Subtraction:
      \begin{equation*}
        \colorspacing
        \colorldlt{\cola} \pm \colorldlt{\colc}
        :=
        \Bigg[
        \begin{matrix}
          \tallcmat{\cola} &
          \tallcmat{\colc}
        \end{matrix}
        \Bigg]
        \begin{bmatrix}
          \smallcmat{\cola} \\
          & \pm \smallcmat{\colc}
        \end{bmatrix}
        % flag of the Netherlands:
        \begin{bmatrix}
          \widecmat{\cola} \\
          \widecmat{\colc}
        \end{bmatrix}
      \end{equation*}
    \item Problem: growing rank
      \begin{equation*}
        \colorspacing
        \colorldlt{\cola} + \colorldlt{\cola}
        :=
        \Bigg[
        \begin{matrix}
          \tallcmat{\cola} &
          \tallcmat{\cola}
        \end{matrix}
        \Bigg]
        \begin{bmatrix}
          \smallcmat{\cola} \\
          & \smallcmat{\cola}
        \end{bmatrix}
        % flag of Austria:
        \begin{bmatrix}
          \widecmat{\cola} \\
          \widecmat{\cola}
        \end{bmatrix}
        %\not\equiv
        %\mathop{\tallcmat{\cola}}
        %\mathop{(\smallcmat{\cola} + \smallcmat{\cola})}
        %\mathop{\widecmat{\cola}}
      \end{equation*}
      $\leadsto$ Column compression techniques
  \end{itemize}
\end{frame}

\subsection{Rosenbrock Method}

\begin{frame}<1>{Rosenbrock Method}
\begin{columns}
\column{0.7\textwidth}
  \begin{block}{General Formulation}
    For the initial value problem (IVP) $\dot x = f(x)$ the method reads
    \begin{equation*}
    \left\{
    \begin{aligned}
      x_{n+1} &:= x_n + \tau \sum_{j=1}^s b_j k_j
      \\
      k_i &:= \begin{multlined}[t]
      f\left( x_n + \tau \sum_{j=1}^{i-1} \alpha_{ij} k_j \right) + \tau \Jac \sum_{j=1}^i \gamma_{ij} k_j
      \\
      \text{for } i = 1, \ldots, s
      ,
      \end{multlined}
    \end{aligned}
    \right.
    \end{equation*}
    where $\Jac := f'(x_n)$ denotes the Jacobian.
  \end{block}
\column{0.3\textwidth}
\bigpicture{3}
\end{columns}
\end{frame}

\begin{frame}<1>{Rosenbrock Method}
\begin{columns}
\column{0.7\textwidth}
  \begin{block}{1st Order Scheme (Linearly Implicit Euler Scheme)}
    For the IVP $\dot x = f(x)$ the method reads
    \begin{equation*}
    \left\{
    \begin{aligned}
      x_{n+1} &= x_n + \tau k_1 \\
      (I - \gamma\tau \Jac) k_1 &= f(x_n)
      .
    \end{aligned}
    \right.
    \end{equation*}
  \end{block}
  \begin{block}{2nd Order Scheme \parencite{Verwer1999}}
    %\vspace*{-\baselineskip}
    %For the IVP $\dot x = f(x)$ the method reads
    \begin{equation*}
    \left\{
    \begin{aligned}
      x_{n+1} &= x_n + \tfrac{3}{2} \tau k_1 + \tfrac{1}{2} \tau k_2 \\
      (I - \gamma\tau \Jac) k_1 &= f(x_n) \\
      (I - \gamma\tau \Jac) k_2 &= f(x_n + \tau k_1) - 2k_1
    \end{aligned}
    \right.
    \end{equation*}
  \end{block}
\column{0.3\textwidth}
\bigpicture{3}
\end{columns}
\end{frame}

\begin{frame}<-4>{Rosenbrock Method}
\begin{columns}
\column{0.7\textwidth}
\begin{itemize}
\item
  Consider the DRE $E^\T \dot X E = \Ricc(X)$.
  \newcommand\U{\alert{\makebox[\widthof{$X_n$}]{$U$}}}
  \begin{align*}
    \Ricc(X_n) &= C^\T C + A^\T X_n E + E^\T X_n A - E^\T X_n BB^\T X_n E
    \\
    \pause
    \Jac(\alert{U}) &= \makebox[\widthof{$C^\T C$}]{$0$}
      + A^\T \U E + E^\T \U A
      \begin{lgathered}[t]
        {} - E^\T \U BB^\T X_n E \\
        {} - E^\T X_n BB^\T \U E
      \end{lgathered} \\
    \pause
    &= (A - BB^\T X_n E)^\T \alert U E + E^\T \alert U (A - BB^\T X_n E)
  \end{align*}
\item
  The Jacobian is a Lyapunov operator!
\pause
\item %TODO: when \leadsto, when \implies?
  $I - \gamma\tau\Jac$ is a Lyapunov operator,
  \ie all Rosenbrock stages are Algebraic Lyapunov Equations (ALEs).
\end{itemize}
\column{0.3\textwidth}
\bigpicture[3]{3}
\end{columns}
\end{frame}

\begin{frame}<1>{1st Order Rosenbrock Scheme}
\begin{columns}
\column{0.7\textwidth}
  \begin{itemize}
    \item
      Linearly implicit Euler scheme
    \item
      \cite{Mena2007}: Formulation for DRE $E^\T \dot X E = \Ricc(X)$
    \item
      \cite{Lang2017}: Formulation for LRSIF $X(t) = LDL^\T$
    \item
      1 ALE per step:
      \begin{equation*}
        \tilde A_n^\T X_{n+1} E + E^\T X_{n+1} \tilde A_n = - GSG^\T
      \end{equation*}
      where
      \begin{align*}
        \tilde A_n &= \gamma\tau(A-BB^\T X_n E) - \tfrac{1}{2} E
        \\
        G &= \begin{bmatrix}
          C^\T & E^\T L
        \end{bmatrix}
        \\
        S &= \begin{bmatrix}
          I & . \\
          . & DL^\T BB^\T LD + \tfrac{1}{\tau} D
        \end{bmatrix}
      \end{align*}
  \end{itemize}
  \vfill
\column{0.3\textwidth}
\bigpicture{4}
\end{columns}
\end{frame}

\begin{frame}<1>{2nd Order Rosenbrock Scheme}
\begin{columns}
\column{0.7\textwidth}
  \begin{itemize}
    \item
      \cite{Verwer1999}
    \item
      \cite{Mena2007}: Formulation for DRE $E^\T \dot X E = \Ricc(X)$
    \item
      \cite{Lang2017}: Formulation for LRSIF $X(t) = LDL^\T$
    \item
      2 ALEs per step:
      \begin{equation*}
      \left\{
      \begin{aligned}
        X_{n+1} &= X_n + \big( 2 - \tfrac{1}{2\gamma} \big) \tau K_1 - \tfrac{1}{2} \tau K_{21} \\
        \hat A_n^\T K_1 E + E^\T K_1 \hat A_n &= -\Ricc(X_n) \\
        \hat A_n^\T K_{21} E + E^\T K_{21} \hat A_n &= -\big( \tau^2 K_1 BB^\T K_1 + \big( 2-\tfrac{1}{\gamma} \big) K_1 \big)
      \end{aligned}
      \right.
      \end{equation*}
      (LRSIF right-hand sides not shown)
%    \item
%      Embedded 1st order method: $\tilde X_{n+1} = X_n + \gamma\tau K_1$
  \end{itemize}
\column{0.3\textwidth}
\bigpicture{4}
\end{columns}
\end{frame}

\subsection{Parareal Method}

\begin{frame}<-2>{Parareal Method}
\begin{bigpicturecols}
  \begin{block}{General Formulation \parencite{Lions2001}}
    For the IVP $\dot u = f(u)$ the method reads
    \begin{equation*}
      \left\{
      \begin{aligned}
        U^0_{n+1} &:= G(U^0_n) \\
        U^{k+1}_{n+1} &:= G(U^{k+1}_n) + F(U^k_n) - G(U^k_n)
      \end{aligned}
      \right.
    \end{equation*}
    where $U_0^0 := u(t_0)$. $U_n^k$ converges to $u(t_n)$ as $k\to\infty$.
  \end{block}
  \begin{itemize}
    \item
      Köhler, Saak, and Lang 2016 (GAMM):
      LRSIF formulation
  % GAMM Annual Meeting
  \begin{equation*}
    U^{k+1}_{n+1}
    =\vphantom{\Bigg[}\parbox{6cm}{$%
    \alt<1>{
      \colorldlt{\cola}
    + \colorldlt{\colb}
    - \colorldlt{\colc}
    }{
    \colorspacing % must be located in respective cell!
    \Bigg[
    \begin{matrix}
      \tallcmat{\cola} &
      \tallcmat{\colb} &
      \tallcmat{\colc}
    \end{matrix}
    \Bigg]
    \begin{bmatrix}
      \smallcmat{\cola} \\
      & \smallcmat{\colb} \\
      && -\smallcmat{\colc}
    \end{bmatrix}
    \begin{bmatrix}
      \widecmat{\cola} \\
      \widecmat{\colb} \\
      \widecmat{\colc}
    \end{bmatrix}
    }$}
  \end{equation*}
    \item
      Speed-up $\approx t_F/t_G$ for large $N$,
      where $0 \leq n \leq N$.
  \end{itemize}
\column{\bigpicturewidth}
\bigpicture{6}
\end{bigpicturecols}
\end{frame}

\begin{frame}<-18>[plain,label=parareal_anim]
  \setlength{\intextsep}{0pt}
  \only<-18>{%
    \setbeamertemplate{caption}[numbered]
    \setlength{\abovecaptionskip}{0pt}
    \renewcommand\thefigure{6.4} % number in thesis
  }
  \frametitle{Parareal Example}
\begin{columns}
\column{0.65\textwidth}
  \begin{figure}
  $U^{k+1}_{n+1} := G(U^{k+1}_n) + F(U^k_n) - G(U^k_n)$
  \foreach \k in {0,1} {%
  \foreach \n [evaluate={\i=int(1+\n+\k*7)}] in {0,...,6} {%
    \includegraphics<\i>[height=0.65\textwidth]{figures/parareal-anim/step-\k-\n.pdf}%
  }}%
  \includegraphics<15>[height=0.65\textwidth]{figures/parareal-anim/step-2-5.pdf}%
  \includegraphics<16>[height=0.65\textwidth]{figures/parareal-anim/step-2-6.pdf}%
  \includegraphics<17>[height=0.65\textwidth]{figures/parareal-anim/step-3-5.pdf}%
  \includegraphics<18>[height=0.65\textwidth]{figures/parareal-anim/step-3-6.pdf}%
  \includegraphics<19>[height=0.625\textwidth]{figures/slides_timeline_simple}% WTF is wrong with the height?!
  % The update is not always visible; for k=3 most take less than 0.001 seconds, while t_F is about
  \caption{\alt<-18>{%
    Parareal method applied to a linear ODE
  }{%
    Timeline diagram for $n \leq N=10$ and $k \leq K = 4$
  }}
  \end{figure}
\column{0.35\textwidth}
  \input{figures/slide_parareal_anim.tex}
\end{columns}
\end{frame}

% Move comparison of dependencies and timeline to its own frame:
\againframe<19>[plain]{parareal_anim}

\section{Results}

\subsection{Numerical Results}

\begin{frame}{Numerical Results}
  \setbeamertemplate{caption}[numbered]
  \begin{columns}
  \column{0.7\textwidth}
  \begin{minipage}[b][0.75\textwidth][c]{\textwidth}
\only<+>{
  \begin{figure}
  \renewcommand\thefigure{7.7} % number in thesis
  \includegraphics[width=\textwidth]{figures/fig_results_parareal.pdf}%
  \caption{Trajectory $X_{1,77}$ and relative error in $K$ for Rail371}
  \end{figure}
}
\only<+>{
  \begin{figure}
  \renewcommand\thefigure{7.8} % number in thesis
  \includegraphics[width=\textwidth]{figures/fig_results_parareal_rank.pdf}%
  \caption{Rank of $X=LDL^\T$ for Rail371}
  \end{figure}
}
\only<+>{
  \begin{figure}
  \renewcommand\thefigure{7.9} % number in thesis
  \includegraphics[width=\textwidth]{figures/fig_timeline_all.pdf}%
  \caption{Timeline chart of parareal method applied to Rail371}
  \end{figure}
}
  \end{minipage}
  \column{0.3\textwidth}
  \begin{block}{Parareal Setup}
    \begin{itemize}
      \item
        450 coarse steps (\SI{100}{\milli\second})
      \item
        100 fine steps per coarse step (\SI{1}{\milli\second})
      \item
        Maximum \#iterations: 10
      \item
        Convergence:
        relative change in~$X$\\ below $371\umach$,\\
        twice in a row,\\
        and all previous stages converged
    \end{itemize}
  \end{block}
  \end{columns}
\end{frame}

\subsection{Parallel Scaling}

\begin{frame}[b,fragile,label=current]{Parallel Scaling}
  \setbeamertemplate{caption}[numbered]
  \begin{columns}[c]
  \column{0.65\textwidth}
  \begin{table}
  %TODO: use hanging captions
  \renewcommand\thetable{7.3} % number in thesis
  \caption{%
    Speed-up and parallel efficiency of parareal method applied to Rail371 using $N=450$ cores.
    (timings in seconds)
  }
  \begin{tabular}{%
    l
    S[table-format=4.2] % par
    S[table-format=6.2] % seq est
    S[table-format=2.2] % speedup
    S[round-precision=3, round-minimum=0.001, table-format=1.3, table-space-text-post=$^{*}$] % efficiency
  }
    \toprule
    Solver &
    {$\tpar$} &
    {$\hattseq$} &
    {$\frac{\hattseq}{\tpar}$} &
    {$\frac{\hattseq}{N\cdot\tpar}$} \\
    \midrule
    \input{tables/speedup450_lr.tex}
    \addlinespace
    \input{tables/speedup450_de.tex}
    \addlinespace
    \input{tables/speedup450_ref.tex}
    \midrule
    \pause
    Rail1357 & 3001.4058759212494 & 22684.368317604065 & 7.557914275969537 & 0.016795365057710083$^{*}$ \\
    \bottomrule
  \end{tabular}
  \end{table}
  \column{0.35\textwidth}
  \begin{block}{Addendum}
  \begin{itemize}
    \item
      Actual runtime of (sequential) Dense 4:

      $\tseq < \SI{86831}{\second}$

      (Slurm job duration)
    \item
      LRSIF 1/1 applied to Rail1357:
      % TODO: goto button for timeline (and back button there)

      \begin{itemize}
        \item
          2 BLAS threads\\ per process
        \item
          $2\times$ round-robin scheduling onto\\
          $P=225$ processes
        \item[{\makebox[\widthof{\usebeamertemplate{itemize item}}][c]{$\ast$}}]
          actual efficiency:
          $2\hattseq/2P\cdot \tpar = \num[round-precision=3]{0.033590730115420166}$
      \end{itemize}

      % $\tpar = \SI{3600}{\second}$ (Slurm job duration)
  \end{itemize}
  \end{block}
  \end{columns}
  \onslide
  \vfill
  \begin{lstlisting}
MY_KIND=dense MY_NF=100 MY_OF=1 MY_OC=1 sbatch -n450 -J de11 par.job
  \end{lstlisting}
\end{frame}


\section{Summary}

\begin{frame}
  \frametitle{Summary}
  foo
\end{frame}

\appendix

\documentclass[
  aspectratio=1610,
]{beamer}

\includeonlyframes{rail1357,current,other}

\usepackage[american]{babel}
\usepackage[utf8]{inputenc}
\usepackage[T1]{fontenc}

% needs https://tex.stackexchange.com/questions/423848/xelatex-xy-and-dejavu-otf#423854
%\usepackage{dejavu-otf} % default Makie font: DejaVu Sans
%\usefonttheme{professionalfonts}

\title{A Low-Rank Parareal Solver for\\ Differential Riccati Equations\\ Written in Julia}
\author{Jonas Schulze}
\institute{Faculty of Mathematics\\ Otto-von-Guericke-Universität Magdeburg}
\date{May 10, 2022}
\subject{subject}

% beamer appearance
\setbeamercolor{block body}{bg=mpi} % for debugging
\setbeamercovered{transparent}
\beamertemplatenavigationsymbolsempty

\newcommand\maketocframe[1][]{%
  \begin{frame}{Outline}
    \tableofcontents[#1]
  \end{frame}
}

\AtBeginSection{%
  \maketocframe[currentsection,currentsubsection]
}

\usepackage[
  style=authoryear,
]{biblatex}
\addbibresource{stuff.bib}

\input{preamble/packages-slides}
\input{preamble/math.tex}
\input{preamble/other.tex}
\input{preamble/abbreviations}
\input{preamble/drawing}

\renewcommand\mathrm\mathsf % fix \umach

\begin{document}

\frame[plain]{\titlepage}
\maketocframe

\section{Motivation}

\everymath{\displaystyle}

\begin{frame}{Motivation}
  \begin{columns}[t,onlytextwidth]
  \column{0.5\linewidth}
  \begin{block}{\strut Optimal Control Problem}
    Consider the \ac{LQR} problem
    \begin{equation*}
      \begin{array}{cl}
        \min_u & \int_{t_0}^{t_f} y^\T y + u^\T u \dt + \tfrac{1}{100} y(t_f)^\T y(t_f) \medskip\\
        \text{s.t.} & \begin{aligned}[t]
          E \dot x &= Ax + Bu \\
          y &= Cx
        \end{aligned}
      \end{array}
    \end{equation*}
    using
    \begin{itemize}
      \item
        state $x(t)\in\R^n$
      \item
        control $u(t)\in\R^m$, $m\ll n$
      \item
        output $y(t)\in\R^q$, $q\ll n$
      \item
        autonomous system matrices $E, A \in\Rnn$, $B\in\R^{n\times m}$, $C\in\R^{q\times n}$
    \end{itemize}
  \end{block}
  \column{0.45\linewidth}
  \begin{block}{\strut Feedback Law}
    The optimal control is given by
    \begin{equation*}
      u(t) = - \underbrace{
        B^\T X(t) E
      }_{
        K\mathrlap{(t)}
      }
      x(t)
    \end{equation*}
    where $X(t)\in\R^{n\times n}$ solves the \ac{DRE}
    \begin{equation*}
      \left\{
        \begin{aligned}
          E\dot X E &= C^\T C + A^\T X E + E^\T X A - E^\T X BB^\T X E \\
          E^\T X(t_0) E &= \tfrac{1}{100} C^\T C
        \end{aligned}
      \right.
    \end{equation*}
  \end{block}
  \end{columns}
\end{frame}

\begin{frame}{Notes}
  \begin{itemize}
    \item
      \enquote{low-rank} versions of a matrix or algorithm refer to \enquote{LRSIF}
  \end{itemize}
\end{frame}

\begin{frame}
  \begin{itemize}
    \item
      For a resolution of \SI{1}{\milli\second} along $[t_0,t_f] = [0, \SI{45}{\second}]$,
      storing $X$ takes ... memory

      $\leadsto$ only store $K := B^\T X E$
    \item
      For moderate size $n=\num{20000}$, a single (dense) $X(t)\in\Rnn$ takes \SI{3.2}{\giga\byte}.

      But: solution usually has low numerical rank \\
      \parencite[e.g.][]{Lang2017,Kuerschner2016,Penzl2000}

      $\leadsto$ \ac{LRSIF}
      \begin{equation}
        \bigmat{X} = \mathop{\tallmat{L}} \mathop{\smallmat{D}} \mathop{\widemat{L^{\smash{\T}}}}
      \end{equation}
    \item
      For a resolution of \SI{1}{\milli\second} along $[t_0,t_f] = [0, \SI{45}{\second}]$,
      and small $n=371$,
      computing a 4th order trajectory sequentially takes about 1 day.

      $\leadsto$ need further parallelization
  \end{itemize}
\end{frame}

\newcommand\tallcmat[1]{\tikz[baseline=-0.5ex]\node[tallmat,fill=#1] {};}
\newcommand\smallcmat[1]{\tikz\node[smallmat,fill=#1] {};}
\newcommand\widecmat[1]{\tikz\node[widemat,fill=#1] {};}
\newcommand\colorldlt[1]{%
  \mathop{\tallcmat{#1}}
  \mathop{\smallcmat{#1}}
  \mathop{\widecmat{#1}}
}

\newcommand\colorspacing{%
  \arraycolsep=3pt
  \def\arraystretch{0.75}
}

\begin{frame}{Low-Rank Symmetric Indefinite Factorization}
  \begin{itemize}
    \item \cite{Benner2009}
    \item Addition/Subtraction:
      \begin{equation*}
        \colorspacing
        \colorldlt{\cola} \pm \colorldlt{\colc}
        :=
        \Bigg[
        \begin{matrix}
          \tallcmat{\cola} &
          \tallcmat{\colc}
        \end{matrix}
        \Bigg]
        \begin{bmatrix}
          \smallcmat{\cola} \\
          & \pm \smallcmat{\colc}
        \end{bmatrix}
        % flag of the Netherlands:
        \begin{bmatrix}
          \widecmat{\cola} \\
          \widecmat{\colc}
        \end{bmatrix}
      \end{equation*}
    \item Problem: growing rank
      \begin{equation*}
        \colorspacing
        \colorldlt{\cola} + \colorldlt{\cola}
        :=
        \Bigg[
        \begin{matrix}
          \tallcmat{\cola} &
          \tallcmat{\cola}
        \end{matrix}
        \Bigg]
        \begin{bmatrix}
          \smallcmat{\cola} \\
          & \smallcmat{\cola}
        \end{bmatrix}
        % flag of Austria:
        \begin{bmatrix}
          \widecmat{\cola} \\
          \widecmat{\cola}
        \end{bmatrix}
        %\not\equiv
        %\mathop{\tallcmat{\cola}}
        %\mathop{(\smallcmat{\cola} + \smallcmat{\cola})}
        %\mathop{\widecmat{\cola}}
      \end{equation*}
      $\leadsto$ Column compression techniques
  \end{itemize}
\end{frame}

\input{content/slides_rosenbrock}
\input{content/slides_parareal}
\input{content/slides_results}

\section{Summary}

\begin{frame}
  \frametitle{Summary}
  foo
\end{frame}

\appendix

\input{appendix/slides}

\end{document}


\end{document}


\end{document}


\end{document}
