\chapter{ADI Reordering Following \citeauthor{Li2002}}
\label{sec:li2002}

\citeauthor{Li2002} \cite{Li2002} propose a reordering of the terms to simplify computations.
The exact motivation will become apparent when applying the ADI to \ac{ALE}.
For an initial value of $x^0 = 0$,
the ADI reads:
\begin{align*}
  x^{k+1}
  &= G_k x^k + B_k b \\
  &= G_k G_{k-1} x^{k-1} + (G_k B_{k-1} + B_k) b \\
  &= G_k G_{k-1} (G_{k-2} x^{k-2} + B_{k-2}b) + (G_k B_{k-1} + B_k) b \\
  &= G_k \cdots G_{k-2} x^{k-2} + (G_k G_{k-1} B_{k-2} + G_k B_{k-1} + B_k) b \\
  &\vdotswithin{=} \\
  &= \underbrace{G_k \cdots G_0 x^0}_0 {} + (G_k \cdots G_1 B_0 + \ldots + G_k B_{k-1} + B_k) b \\
\intertext{%
  By \autoref{thm:adi:permutation} we may introduce a permutation $\sigma$ of $\Set{0,\ldots,k}$.
  Thus:
}
  &= (G_{\sigma(k)} \cdots G_{\sigma(1)} B_{\sigma(0)} + \ldots + G_{\sigma(k)} B_{\sigma(k-1)} + B_{\sigma(k)}) b \\
\intertext{%
  Choosing $\sigma = \sigma_k : i \mapsto k-i$,
  and exploiting that $G_* B_* = B_* G_*$ commute,
  yields:
}
  &= (G_0 \cdots G_{k-1} B_k + \ldots + G_0 B_1 + B_0) b \\
  &= (B_k G_{k-1} \cdots G_0 + \ldots + B_1 G_0 + B_0) b \\
\intertext{%
  Further utilizing $G_k := M_k^{-1} N_k = N_k M_k^{-1}$ and $B_k := M_k^{-1}$ leads to:
}
  &= \big( M_k^{-1} (N_{k-1} M_{k-1}^{-1}) \cdots (N_0 M_0^{-1}) + \ldots + M_1^{-1} (N_0 M_0^{-1}) + M_0^{-1} \big) b \\
  &= \big( (M_k^{-1} N_{k-1}) \cdots (M_1^{-1} N_0) M_0^{-1} + \ldots + (M_1^{-1} N_0) M_0^{-1} + M_0^{-1} \big) b \\
  &= \underbrace{
    (M_k^{-1} N_{k-1}) \cdots (M_1^{-1} N_0) M_0^{-1} b
  }_{v^k}
  + \underbrace{
    \big( \ldots + (M_1^{-1} N_0) M_0^{-1} + M_0^{-1} \big) b
  }_{x^k}
\end{align*}
Regrouping reveals:
\begin{equation*}
\left\{
\begin{aligned}
  x^{k+1} &= x^k + v^k &
  x^0 &= 0 \\
  v^k &= (M_k^{-1} N_{k-1}) v^{k-1} &
  v^0 &= M_0^{-1} b
\end{aligned}
\right.
\end{equation*}
